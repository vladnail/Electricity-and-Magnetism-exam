\section*{27. Магнитное поле. Сила Ампера. Закон Био – Савара. Сила Лоренца.
Движение заряда в магнитном поле. Дрейф в скрещенных полях.}
 
\subsection*{Магнитное поле}

\imc[0.6\textwidth]{50.png}

Известно что между проводниками, по которым протекают электри-
ческие токи, возникают силы взаимодействия. Точно так же, как в элек-
тростатике, где введение электрического поля позволяет удобным обра-
зом описать взаимодействие статических зарядов, здесь полезно ввести
понятие \textit{магнитного поля}.

\textit{Магнитное поле в вакууме:}

\imc[0.4\textwidth]{55.png}

\[
d\vec{F_{12}}= \frac{I_1 I_2 }{c^2} \frac{[d\vec{l_2}\times [d\vec{l_1}\times \vec{r}]]}{r_{3}}  
\]

Постулируем, но понимаем, что: $dF_{12}\neq dF_{21}$

\subsection*{Сила Ампера и Закон Био – Савара}

\[
dF_{12}=\frac{I_2}{c}[d\vec{l}\times \vec{H}]-\textit{сила Ампера}  
\]

Где $\vec{H}:$

\[
dH=\frac{I}{c} \frac{[dl \times r]}{r^3} - \textit{закон Био-Савара}  
\]

\subsection*{Сила Лоренца}

\imc[0.4\textwidth]{56.png}

\textit{Сила Лоренца}-это сила, которая дейтвует на заряд со стороны магнтиного поля. 

\[
    dF_=\frac{I}{c}[d\vec{l}\times \vec{H}],\vec{j}=qn\vec{v}
\]

\[
Id\vec{l}=\vec{j}dV \Rightarrow Idl=jSdl
\]

\[
dF=\frac{1}{c}[Id\vec{l}\times \vec{H}]=\frac{1}{c}[\vec{j}dv \times \vec{H}]=\frac{1}{c} [dV\cdot nq\vec{v}\times \vec{H} ] \Rightarrow 
\]

\[
\Rightarrow\text{для единичного заряда: }\vec{F}=\frac{q}{c}[\vec{v}\times \vec{H}] 
\]

Если есть $\vec{H}\text{ и }\vec{E}\text{ , то: }\boxed{\vec{F}=q \bigg[\vec{E}+\frac{1}{c}[\vec{v} \times \vec{H}]\bigg] }$

\newpage

\subsection*{Движение заряда в магнитном поле}

\imc[0.35\textwidth]{57.png}

\[
\vec{F}=\gamma m\overset{.}{\vec{v}}=\frac{q}{c}[\vec{v}\times \vec{H}] \Rightarrow \overset{.}{\vec{v}}=\left[ \left( -\frac{q\vec{H}}{\gamma mc}  \right) \times \vec{v} \right]
\]