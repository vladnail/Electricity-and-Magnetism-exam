
\section{Уравнение Лапласа. Разделение переменных в уравнении Лапласа в
сферической системе координат.}

\subsection*{Уравнение Лапласа(повтор)}

\subsection*{Разделение переменных в уравнении Лапласа в
сферической системе координат}

Пусть $\varphi(r,\theta,\alpha)=R(r)\cdot Y(\theta)$

\fc{\Delta \varphi(r,\theta,\alpha)=0\Rightarrow \underset{=l(l+1)}{\frac{1}{R}\frac{d}{dr} \left( r^2 \frac{dR}{dr} \right)}+\underset{=-l(l+1)}{\frac{1}{Y \sin \theta}\frac{d}{d\theta} \left( \sin \theta \frac{dY}{d\theta} \right)}=0}

При $R(r)\varpropto r'$

или $R(r)\varpropto \frac{1}{r^{l+1}}$

\fc{\frac{1}{R}(r^2R')'=C} 

ищем решение в виде $R(r)\varpropto r^l$

\fc{\frac{1}{R}(r^2R')'=\underset{=-(l'+1)}{l}\cdot \underset{=(-l'-1+1)=(l'+1)l'}{(l+1)}} 

При этом $R(r)\varpropto \frac{1}{r^{l+1}}$ удолетвор. уравнению с той же С 

(замена $l'=-(l+1)$) 
