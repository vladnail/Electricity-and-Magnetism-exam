\section{Энергия магнитного поля. Плотность энергии магнитного поля.}

\[
\vec{B}=\mathrm{rot}\vec{A} \qquad \mathrm{rot}\vec{H}=\frac{4\pi}{c}\vec{j}
\]

Для линейной среды:

\begin{gather*}
    W=\frac{1}{2c}\iiint \vec{A}\vec{j}dV=\frac{1}{8\pi}\iiint \vec{A}[\grad \times \vec{H}]dV=\frac{1}{8\pi}\iiint \grad[\overset{\downarrow}{\vec{H}}\times \vec{A}]dV= \\
    =\frac{1}{8\pi}\iiint \left( \grad[\overset{\downarrow}{\vec{H}}\times \overset{\downarrow}{\vec{A}}]-\grad [\vec{H}\times \overset{\downarrow}{\vec{A}}] \right)dV= \\
    =\frac{1}{8\pi} \cancelto{0}{\underset{\mathbb{S}_{\infty }}{\oiint}[\vec{H}\times \vec{A}]d\vec{S} }+\frac{1}{8\pi}\iiint\vec{H}[\grad \times \vec{A}]dV=\iiint \frac{\vec{H}\vec{B}}{8\pi}dV  
\end{gather*}

Плотность энергии магнитного поля: \( \omega= \frac{\vec{H}\vec{B}}{8\pi} \Rightarrow \delta W= \iiint \delta\omega dV \)

Аналогично для нелинейной среды:

\[
d\omega= \frac{\vec{H}d\vec{B}}{4\pi} \Rightarrow \delta W = \iiint \delta \omega dV
\]