\section{Ферромагнетизм. Гистерезис. Коэрцитивная сила. Остаточное поле.}

1) Обменное взаимодействие \\
2) Энергия магнитного поля \\
3) Ось легкого намагничивания \\
4) Внешнее магнитное поле 

1. Спины соседних электронов направлены одинаково т.к. это энергетически выгодно (наименьшая энергия) \\
2. Чем больше поле тем больше энергии, т.е. это энергетически выгодно 

1,2 \( \Rightarrow \) образование доменов: 

\imc[0.31\textwidth]{78.png}

\newpage

Движение доменов стенки:

\imc[0.68\textwidth]{79.png}

3. Поликристаллы состоят из большого числа кристаллических зерен, каждый из которых имеет собственную ось легкого намагничивания. Совокупность этих выделенных осей в поликристаллах приводит к тому, что в отсутствие внешнего поля суммарная намагниченность материала равна нулю. При приложении внешнего магнитного поля домены начинают перестраиваться, ориентируясь вдоль поля, что приводит к росту намагниченности. \\
4. \( u=-\vec{m}\vec{H} \) \\
5. Примеси в ферромагнетике создают препятствия для движения доменных стенок. При малых значениях внешнего магнитного поля доменные стенки практически не двигаются, так как энергия поля недостаточна для преодоления этих препятствий. Однако при увеличении поля доменные стенки начинают двигаться скачками, что приводит к явлению, известному как прыжки Баркгаузена. 
Из-за этого возникает петля Гистерезиса.

\imc[0.55\textwidth]{80.png}

\( H_c \) - коэрцитивная сила

\( B_c \) - остаточная намагниченость 

Минорная петля гистерезиса — это петля намагничивания, получаемая при изменении внешнего магнитного поля в пределах, меньших, чем насыщение материала.

Если максимальное значение внешнего магнитного поля ( H ) не достигает насыщения ферромагнетика, намагниченность ( M ) изменяется в меньшем диапазоне, формируя минорную петлю(самая маленькая петля) внутри основной  петли гистерезиса.

Связь между намагниченостью и магнитными полями: \( \vec{B}=\vec{H}+4\pi \vec{M} \) 

6. Температура Кюри: пропадает феррамагнетизм при повышении темепературы.

\imc[0.55\textwidth]{81.png}