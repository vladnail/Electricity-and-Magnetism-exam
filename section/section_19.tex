\section*{19. Энергия электрического поля в диэлектрике.}

Линейная среда $(\vec{D}=\varepsilon\vec{E},\vec{P}=\chi\vec{E})$, все похоже на вакуум.

\imc[0.3\textwidth]{39.png} 

\fc{\delta W=\varphi_i dq_i\Rightarrow W=\frac{1}{2}\varphi_i q_i}

\fc{\widetilde{q_i}=\alpha q_i\text{ и }\widetilde{\varphi_i}=\alpha\varphi_i\text{, где } 0\leq\varphi\leq1}

Перепишем энергию:

\fc{W=\int dW=\int \widetilde{\varphi_i}d\widetilde{q_i}=\int_0^1\alpha\varphi_i q_id\alpha=\frac{1}{2}\varphi_i q_i}
\newpage
\kr{Доказательство:}

\fc{W=\underset{V}{\iiint} \frac{\vec{E}\vec{D}}{8\pi}dV=\frac{1}{8\pi}\underset{V}{\iiint}(-\grad\overset{\downarrow}{\varphi})\vec{D}dV=}

\fc{\text{Примеменим:}(\grad(\varphi\vec{D}))=\vec{D}\grad\varphi+\varphi\grad\vec{D}}

\fc{=-\frac{1}{8\pi}\underset{V}{\iiint}\grad(\varphi\vec{D})dV+\frac{1}{8\pi}\underset{V}{\iiint}\underset{\rightarrow0}{\varphi\grad\vec{D}}dV=}

\fc{=\underset{i}{\Sigma}\underset{S_i}{\oiint}\frac{1}{8\pi}\varphi\vec{D}d\vec{S}-\underset{S_{\infty}}{\oiint}\varphi\underset{\rightarrow0(\propto1)}{\vec{D}}d\vec{S}=\frac{1}{8\pi}\underset{i}{\Sigma}\varphi_i\underset{S_i}{\oiint}\vec{D}d\vec{S}=\frac{1}{2}\underset{i}{\Sigma}\varphi_i q_i =\frac{1}{2}\varphi_i q_i} 

\kr{Доказано.}

Нелинейная среда: $\delta W=\varphi_i \delta q_i \neq> W=\frac{1}{2}\varphi_i dq_i$

В предыдущем пункте всюду меняем:

\fc{W=\delta W,q_i\rightarrow \delta q_i,\vec{D}=\delta\vec{D}}
\fc{\delta W=\iiint \frac{\vec{E}\delta\vec{D}}{8\pi}dV}