\section*{8. Проводники в электрическом поле. Теорема единственности.}

\subsection*{Проводники в электрическом поле}

Очень похоже (скорее всего есть одно и тоже) на вопрос: поле вблизи поверхности металлов, ну рассмотрим повторно?

\imc[0.4\textwidth]{13.png}

\newpage

Если в проводнике имеется электрическое поле, то по нему течёт ток. Следовательно, для электростатических явлений электрическое поле внутри проводника $E_{i}\equiv0$ отсюда плотность заряда: 

\fc{\rho_i=\frac{1}{4\pi}\mm{div}\vec{E_i}\equiv0}

В этой связи говорят, что проводник квазинейтрален. Таким образом,
заряды на проводнике могут размещаться только на его поверхности,
причем поверхностная плотность зарядов связана с полем $vec{E}$ вне про-водника через граничное условие для $E_n$.

Если пространство вне проводника свободно от зарядов, то здесь поле 
$\vec{E}=-\mm{grad}\varphi$ и $\varphi$ удовлетворяет уравнению Лапласа.

Из граничных условий мы получаем что:

\fc{\vec{E}_{n}=4\pi \sigma \text{ , } \vec{E}_{\tau}=0 .}

Заметим, что поле подходит к поверхности проводника по нормали, т.е. поверхность
проводника является эквипотенциалью. Это естественно, так как в проводнике потенциал постоянен из-за $\vec{E_i}=0$

\subsection*{Теорема единственности}

\kr{Условия теоремы:}

1)На каждом проводнике задан либо потенциал, либо заряд,

2)В $V$ нету зарядов;

$\Rightarrow \exists$ единственное решение уравнения Пуассона вида:

\fc{\vec{E}=-\grad\varphi }

\kr{Доказательсвто}

Пусть $\vec{E_1}=-\grad\varphi_1$ и $\vec{E_2}=-\grad\varphi_2$. Достаточно доказать, что:

\fc{\underset{V}{\iiint}|\vec{E_2}(\vec{r})-\vec{E_1}(\vec{r})|^2dV=0}

\fc{\vec{E}:=\vec{E_2}-\vec{E_2} \text{ ; } \varphi:=\varphi_2-\varphi_1 \text{ ; } \vec{E} =-\grad\varphi_2+\grad\varphi_1=-\grad\varphi   }

Всюду в $V$ $\Delta\varphi_1=0$ и $\Delta\varphi_2=0\Rightarrow \Delta\varphi=0$

Рассмотрим выражение: $\grad(\varphi\grad\varphi)=(\grad\varphi)^2+\underset{\rightarrow0}{\varphi\grad\varphi}=(\grad\varphi)^2$

\fc{{\iiint}|\vec{E_2}(\vec{r})-\vec{E_1}(\vec{r})|^2dV={\iiint}|\vec{E}|^2dV=\iiint(\grad\varphi)^2dV=\iiint\grad(\varphi\grad\varphi)dV=}

\fc{=\underset{V}{\iiint}\mm{div}(\varphi\grad\varphi)dV= \underset{\rightarrow0(\propto \frac{1}{r})}{\underset{S_\infty}{\oiint}\varphi\grad\varphi d\vec{S}}-\underset{i}{\Sigma}\underset{S_i}{\oiint}\varphi\grad\varphi d\vec{S}=\underset{i}{\Sigma}\varphi_i \underset{S_i}{\oiint}(-\grad\varphi) d\vec{S}=}

\fc{\underset{i}{\Sigma}(\varphi_{2i}-\varphi_{1i})\underset{S_i}{\oiint}(\vec{E_2}-\vec{E_1})d\vec{S}=\underset{i}{\Sigma}(\varphi_{2i}-\varphi_{1i}) \left[\underset{S_i}{\oiint}\vec{E_2}d\vec{S}-\underset{S_i}{\oiint}\vec{E_1}d\vec{S}\right]=}

\fc{=4\pi \underset{i}{\Sigma}\underset{(1)}{(\varphi_{2i}-\varphi_{1i})}\underset{(2)}{(q_{2i}-q_{1i})}=0}

По условию теоремы либо (1) = 0, либо (2)=0

\kr{Доказано}
