\section*{9. Метод изображения для решения задач электростатики на примере плоской и сферической границ раздела проводника и непроводящего пространства.}

\kr{Плоская граница}

Точечный заряд  $q$ , находящийся на расстоянии  $h$  от проводящего полупространства. Определить поле в свободном полупространстве и на этой основе — плотность зарядов, индуцированных зарядом  $q$  на поверхности проводника.

\imc[0.6\textwidth]{14.png}

В проводящем полупространстве поле равно нулю, постоянный потенциал можно принять за ноль, будем искать поле только в области $z>0$ с выкинутой точкой. Искомое поле удовлетворяет уравнению Лапласа:

\fc{\Delta \varphi=0}

и граничным условиям

\fc{\varphi|_{z=0}=0 \text{ , } \underset{S_\varepsilon}{\oint}E_n dS=4 \pi Q}

\newpage

где $S_\varepsilon $ сфера малого радиуса с центром в точке
расположения заряда $q$

В проводящем полупространстве будет наводится заряд $q'=-q$. Тогда потенциал и электрическое поле, созданные зарядом $q$ фиктивным зарядом $q'$
,в правом полупространстве создают искомое поле:

\fc{\varphi=\frac{q}{r}-\frac{q}{r_1}}

Действительно, эта функция удовлетворяет уравнению Лапласа в
области $z>0$ как потенциал двух точечных зарядов, лежащих вне области. Во-вторых, $\varphi |_{z=0}=0$, так как для точек плоскости $r$ и $r_1$ равны.
В-третьих, поле, созданное зарядом $q'$, через поверхность $S\varepsilon$ создает
поток, равный нулю (по теореме Гаусса), а поле от точечного заряда
$q$ обеспечивает выполнение соответствующего граничного условия.

\fc{\varphi(\vec{r}) =
\left\{
\begin{aligned}
\frac{q}{|\vec{r}|}-\frac{q}{|\vec{r_1}|}\text{ , }z\geq0 \\
 0	\text{, z<0}	\\
\end{aligned}
\right.}

и 

\fc{\vec{E}(\vec{r}) =
\left\{
\begin{aligned}
\frac{q}{|\vec{r}|^2}\frac{\vec{r}}{|\vec{r}|}-\frac{q}{|\vec{r_1}|^2}\frac{\vec{r_1}}{|\vec{r_1}|}\text{ , }z\geq0 \\
 0	\text{, z<0}	\\
\end{aligned}
\right.}

Таким образом, задача решена.



\kr{Для сферической границы}

Заряд $q$ на расстоянии $l+x$ от центра шара, а потенциал шара принят равным нулю. 

\imc[0.55\textwidth]{15.png}

Искомый потенциал в
произвольной точке P вне шара в этом случае:

\fc{\varphi(P)=\frac{q}{r}+\frac{q'}{r'}}

где $q'=-q\frac{a}{l}$

\newpage

 Решение удовлетворяет уравнению Лапласа в своей области определения, имеет нужную особенность вблизи точечного заряда q и удовлетворяет граничным условиям ($r_0/r_0'=l/a$), обращаясь в нуль.
 
Рассмотрим второй вариант — точечный заряд $q$ рядом с шаром,
несущим заряд $Q$ (при этом постоянный потенциал шара не определен). В этом случае к существующей системе зарядов $q,q'$ необходимо добавить фиктивный заряд, расположенный в центре шара:

\fc{q''=Q-q'=Q-q\frac{a}{l}}

тогда 

\fc{\varphi(P)=\frac{q}{r}+\frac{q'}{r'}+\frac{q''}{r_*}}

потенциал шара при этом:

\fc{\varphi(P)=\varphi|_S =\frac{q}{r_0}+\frac{q'}{r'_0}+\frac{q''}{a}\Rightarrow \varphi|_0=\frac{q''}{a}=\frac{Q}{a}+\frac{q}{l}}

Таким образом, задача решена.