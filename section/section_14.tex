\section{Электрическая ёмкость. Матрица ёмкостных коэффициентов, её симметричность.}

\subsection*{Электрическая ёмкость}

Конденсатор из двух проводников:

\imc[0.4\textwidth]{27.png}

\fc{C=\frac{|q|}{\varphi_1-\varphi_2}}

Уединенный конденсатор:

\imc[0.2\textwidth]{28.png}

\fc{C=\frac{q}{\varphi_1-\varphi_2}}

Энергия кондесатора:

\fc{W=\frac{q^2}{2C}=\frac{CU^2}{2}=\frac{1}{2} \Sigma q_i\varphi_i  }

\newpage

\subsection*{Матрица емкостных коэффициентов, её симметричность}

\imc[0.4\textwidth]{26.png}

\fc{\begin{pmatrix}
\varphi_1 \\
\varphi_2 \\
\vdots \\
\varphi_i \\
\vdots
\end{pmatrix}
=
\renewcommand{\arraystretch}{1.8} % Увеличивает вертикальные интервалы в матрице
\setlength{\arraycolsep}{1.2em}   % Увеличивает горизонтальные интервалы между столбцами
\begin{pmatrix}
   &   &   \\
   & \overset{\wedge}{S} &   \\
   &   &  
\end{pmatrix}
\quad
\renewcommand{\arraystretch}{1.} 
\setlength{\arraycolsep}{1.2em}   
\begin{pmatrix}
q_1 \\
q_2 \\
\vdots \\
q_i \\
\vdots
\end{pmatrix}
\quad
}

или в другом виде:

\fc{\varphi_{i}=S_{ij}q_{j}}

где $S_{ij}$-\kr{матрица потенциальных коэффициентов,}

\fc{q_{i}=C_{ij}\varphi_j}

где $С_{ij}$-\kr{матрица емкостных коэффициентов,}$S_{ij}^{-1}=C_{ij}$-симметричны.

Свойства матриц:

\fc{dW=\underset{i}{\Sigma}\varphi_i dq_i=\varphi_i dq_i}
и
\fc{W=\frac{1}{2}\varphi_i q_i=dW=\frac{1}{2}\varphi_i dq_i+\frac{1}{2}q_id\varphi_i}
\fc{\Downarrow}

\fc{\varphi_idq_i=q_id\varphi_i}

где $\varphi_i=S_{ij}q_j$

\fc{0=S_{i j} q_j d q_i-q_i S_{i j} d q_i=S_{i j} q_j  d q_i-q_j S_{j i} d q_i=(S_{i j}-S_{ji})q_idq_j}
получаем:

\fc{S_{ij}=S_{ji}}
\fc{\Downarrow}
\fc{C_{ij}=C_{ji}}

это справедливо $\forall q_i$ и $\forall dq_i$.