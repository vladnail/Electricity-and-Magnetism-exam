\documentclass[a4paper,12pt]{article}
\usepackage[utf8]{inputenc}
\usepackage[russian]{babel}
\usepackage{physics}
\usepackage{tikz}
\usepackage{wrapfig} % Для обтекания текста
\usepackage{amsmath, amssymb}
\usepackage{graphicx} % Для работы с изображениями
\usepackage{geometry}
\geometry{top=2cm, bottom=2cm, left=2.5cm, right=2.5cm}

% Увеличиваем размер основного текста
\renewcommand{\normalsize}{\fontsize{14}{17}\selectfont}

% Команда для курсива
\newcommand{\kr}[1]{\textit{#1}}

% Команда для вставки математического выражения
\newcommand{\fc}[1]{\[#1\]}

% Команда для вставки математического
\newcommand{\mm}[1]{\mathrm{#1}}

%Настраиваемый вид замкнутого тройного интеграла(при умножении, в дроби)
\newcommand{\oiiint}{%
  \mathchoice%
    {\mathop{\raisebox{0.2ex}{\scalebox{1.5}[1]{\(\bigcirc\)}}}%
     \!\!\!\!\!\!\!\!\!\!\!\!\!\!\int\!\!\!\int\!\!\!\int}% Displaystyle
    {\mathop{\raisebox{0.2ex}{\scalebox{1.2}[1]{\(\bigcirc\)}}}%
     \!\!\!\!\!\!\!\!\!\!\!\!\int\!\!\!\int\!\!\!\int}% Textstyle
    {\mathop{\raisebox{0.1ex}{\scalebox{1}[1]{\(\bigcirc\)}}}%
     \!\!\!\!\!\!\!\!\!\!\int\!\!\!\int\!\!\!\int}% Scriptstyle
    {\mathop{\raisebox{0.05ex}{\scalebox{0.8}[1]{\(\bigcirc\)}}}%
     \!\!\!\!\!\!\!\!\!\int\!\!\!\int\!\!\!\int}% Scriptscriptstyle
}

%Настраиваемый вид замкнутого двойного интеграла(при умножении, в дроби)
\newcommand{\oiint}{%
  \mathchoice%
    {\mathop{\raisebox{0.1ex}{\scalebox{1.7}[0.7]{\(\bigcirc\)}}}%
     \!\!\!\!\!\!\!\!\!\!\!\int\!\!\!\int}% Displaystyle
    {\mathop{\raisebox{0.3ex}{\scalebox{1.}[0.5]{\(\bigcirc\)}}}%
     \!\!\!\!\!\!\!\int\!\!\!\int}% Textstyle
    {\mathop{\raisebox{0.1ex}{\scalebox{1}[0.8]{\(\bigcirc\)}}}%
     \!\!\!\!\!\!\!\!\!\int\!\!\!\int}% Scriptstyle
    {\mathop{\raisebox{0.05ex}{\scalebox{0.8}[0.8]{\(\bigcirc\)}}}%
     \!\!\!\!\!\!\!\!\int\!\!\!\int}% Scriptscriptstyle
}

% уменьшенный и сдвинутый градиент
\newcommand{\gradd}{\hspace{0.2cm}\scriptsize \nabla}   

  % Команда для изображения в центре
\newcommand{\imc}[2][0.7\textwidth]{%
    \begin{figure}[h!]
        \centering
        \includegraphics[width=#1]{#2}
    \end{figure}%
}

% Команда для изображения слева
\newcommand{\iml}[2][0.7\textwidth]{%
    \begin{figure}[h!]
        \raggedright
        \includegraphics[width=#1]{#2}
    \end{figure}%
}


% Команда для увеличения размера формул
\newcommand{\ds}[1]{\displaystyle #1}

%%%%%%%%%%%%%%%%%%%%%%%%%%%%%%%%%%%%%%%

\begin{document}

% Титульный лист
\begin{titlepage}
    \begin{center}
        \textbf{\large Министерство науки и высшего образования}\\
        \textbf{\largeРоссийской Федерации} \\
        \textbf{\large Федеральное государственное автономное образовательное
учреждение высшего образования} \\
        \textbf{\large «Новосибирский государственный университет» } \\
        \vspace{1em}
        \textbf{\large Физический факультет} \\
        \vspace{5em}
        \textbf{\Large Дисциплина: Электромагнетизм} \\
        \vspace{2em}
        \textbf{\Large Зимняя сессия } \\
        \vspace{25em}
        \textbf{\large Приговор будет исполнен: 11.01.2025}\\
        \vspace{1em}
        \textbf{\kr {Я не несу ответственности за возможные ошибки или некорректность предоставленных ответов на билеты. Используйте их только как вспомогательный материал и обязательно сверяйтесь с официальными источниками.}  }
    \end{center}
\end{titlepage}

\newpage

% Основной текст
\section*{Ответы на вопросы билета}

\section*{1. Закон Кулона. Напряжённость электрического поля. Принцип
суперпозиции.Поток электрического поля. Теорема Гаусса.}

\subsection*{Закон Кулона}

Это — экспериментально установленный закон силового взаимодействия двух
точечных заряженных тел, неподвижных относительно рассматриваемой системы
отсчета, согласно которому:

\[\vec{F_k}=\frac{q_1q_2}{r_{12}^2}\frac{\vec{r_{12}}}{r_{12}}\]

\imc[0.5\textwidth]{2.png} 

Введем понятие напряженности:

\[\vec{E}_1(\vec{r}_2) = \frac{q_1}{r_{12}^2} \frac{\vec{r}_{12}}{r_{12}}\]

тогда силу Кулона можно перезаписать в виде:

\[\vec{F}_{12} = q_2 \vec{E}_1(\vec{r}_2)\]

\newpage


\subsection*{Напряжённость электрического поля}

В общем виде напряженность имеет вид:
\[\vec{E}(\vec{r}) = \frac{q}{|\vec{r} - \vec{r}_0|^2} \frac{\vec{r} -
\vec{r}_0}{|\vec{r} - \vec{r}_0|}\]

\imc[0.5\textwidth]{1.png} 

\subsection*{Принцип суперпозиции}

Электрическое поле от системы зарядов равно сумме электрических полей от её
составляющих:

\fc{\vec{E}(\vec{r}) = \sum_i \vec{E}_i(\vec{r}) = \sum_i \frac{q_i}{|\vec{r} -
\vec{r}_i|^2}\frac{ \vec{r} - \vec{r}_i}{|\vec{r} - \vec{r}_i|}}

\subsection*{Поток электрического поля}

Если у нас имеется некоторая конечная поверхность S, то поток
через эту поверхность вычисляется как поверхностный интеграл

\fc{\text{Ф}= E_ndS}

\subsection*{Теорема Гаусса}

\kr{Теорема Гаусса:}Поток вектора $\vec{E}$ через любую замкнутую
поверхность определяется суммарным зарядом Q, находящимся внутри этой
поверхности, и равняется 4$\pi$Q:

\fc{\oint_SE_nds=4\pi Q}

\newpage


\section*{2. Дивергенция электрического поля. Распределённый заряд. Основное
уравнение электростатики, его общее решение в безграничном пространстве}

\subsection*{Дивергенция электрического поля}

Вспомним теорему Гаусса для потока $\vec{E}$ через замкнутую площадь S

\fc{\underset{\delta V}{\oiint}\vec{E}d\vec{S}=4\pi Q= \underset{V}{\iiint}
4\pi \rho dV }
а по теореме Остроградского-Гаусса 

\fc{\underset{\delta V}{\oiint}\vec{E}d\vec{S}=\underset{V}{\iiint}div
\vec{E}d\vec{V}}

следует что для $ \forall V$ :

\fc{\underset{V}{\iiint}div \vec{E}d\vec{V}=4\pi Q= \underset{V}{\iiint} 4\pi
\rho dV\Rightarrow div\vec{E}=4\pi \rho}  

\subsection*{Распределённый заряд}

Объемная плотность заряда: 

\fc{dq\overset{df}{=}\rho dV}

Поверхностная плотность:

\fc{dq\overset{df}{=}\sigma dS}

Линейная плотность:

\fc{dq\overset{df}{=}\kappa dl}

\subsection*{Основное
уравнение электростатики, его общее решение в безграничном пространстве}

В конечной области пространства с плотностью заряда $\rho(\vec{r})$, по
принципу суперпозиции скалярный потенциал этих зарядов равен:

\fc{\varphi (\vec{r})=\int \frac{\rho(\vec{r'})dV'}{|\vec{r}-\vec{r'}|}}

\imc[0.5\textwidth]{4.png} 

Представление потенциала в виде интеграла по объему, занятому зарядами, часто
называют частным решением уравнения Пуассона.

Для задачи с точечными зарядами интегральная форма не подойдёт, перейдём к
сумме. Введём функцию Дирака $\delta$, она задается следующими условиями:

1) при всех $\vec{r} \neq 0$ $\delta(\vec{r})=0$ ;

2) в точке $\vec{r}\neq 0$ имеем $\delta(\vec{r})=\infty$ ;

3) интеграл по всему пространству $\int\delta(\vec{r})dV=1$

4) $\int f(\vec{r})\delta(\vec{r}-\vec{r_0})dV=f(\vec{r_0})$

где $f(\vec{r})$ - произвольная непрерывная функция, $\vec{r_0}$
радиус-вектор некоторой фиксированной точки.

Объёмную плотность заряда расположенного в точке $\vec{r}=\vec{r_0}$ можно
перезаписать:
\fc{\rho(\vec{r})=q\delta(\vec{r}-\vec{r_0})}

подставляем в предыдущую формулу

\fc{\varphi (\vec{r})=\int
\frac{q\delta(\vec{r}-\vec{r_0})dV'}{|\vec{r}-\vec{r'}|}=\frac{q}{|\vec{r}-\vec
{r'}|}}

\newpage


\section*{3. Циркуляция и ротор электрического поля. Теорема Стокса. Электрический
потенциал. Работа электрического поля. Потенциал точечного заряда.}

\subsection*{Циркуляция и ротор электрического поля}

Циркуляция векторного поля $\vec{E}$ вдоль контура L

\fc{\underset{L}{\int} \vec{E} d\vec{l}} 

а по замкнутому контуру

\fc{\underset{L}{\oint} \vec{E} d\vec{l}=0} 

или в дифференциальной форме 

\fc{\mm{rot}\vec{E}=0}

\imc[0.5\textwidth]{5.png} 

Как следствие из теоремы о циркуляции $\vec{E}$ работа при перемещении
заряда из одной точки поля в другую не зависит от формы траектории движения.

\newpage


\subsection*{Теорема Стокса}

\fc{\underset{\delta
S}{\oint}\vec{E}d\vec{l}=\underset{S}{\iint}\mm{rot}\vec{E}d\vec{S}}

\imc[0.5\textwidth]{6.png}

\fc{(\mm{rot}\vec{E})_{\vec{n}'}=\frac{\mm{rot}\vec{E}}{\frac{1}{\mm{cos}\theta
}}}

где $\frac{1}{\mm{cos}\theta}=\frac{dS_{\vec{n}'}}{dS_{\vec{n}}}$, получим

\fc{(\mm{rot}\vec{E})_{\vec{n}'}=\mm{rot}\vec{E}\cdot\mm{cos}\theta} 


\subsection*{Электрический потенциал}

Рассмотрим скаляроное поле

\imc[0.5\textwidth]{7.png}

\fc{\varphi(\vec{r})\overset{df}{=}\int_{\vec{r}}^{\vec{r_0}}\vec{E}d\vec{l}}

Чтобы определение было корректным, нужно чтобы этот интеграл не зависел от
формы $L$.

\kr{Доказательство:}

\fc{\mm{rot}\vec{E}=0 \Rightarrow \forall L
\underset{L}{\oint}\vec{E}d\vec{l}=0}

Запишем выражение при обходе $L-L'$ - сначала идем по контуру $L$, а
потом обратно по контуру $L'$ : 

\fc{\underset{L-L'}{\oint}\vec{E}d\vec{l}=0=\underset{L}{\oint}\vec{E}d\vec{l}
-\underset{L'}{\oint}\vec{E}d\vec{l}=0}

Такие поля называются потенциальными.

\kr{Доказано.}

Еще свойства потенциала: 

\imc[0.35\textwidth]{8.png}
   
\fc{\varphi(\vec{r})=\int \vec{E}(\vec{r})d\vec{r} + \underset{L}{\int}
\vec{E}d\vec{l} }

\fc{\text{и}}

\fc{\varphi(\vec{r}+d\vec{r_0})=\underset{L}{\int} \vec{E}d\vec{l}}
 
отсюда получим

\fc{\varphi(\vec{r}+d\vec{r_0})-\varphi(\vec{r})=-\vec{E}(\vec{r})d\vec{r}}

\fc{\text{так же используем}}

\fc{d\varphi=d\vec{r}\gradd \varphi }

отсюда получим

\fc{\forall d\vec{r} , \vec{E}(\vec{r})d\vec{r}=-\grad \varphi d\vec{r} \Rightarrow \vec{E}=-\mm{grad} \varphi=-\grad \varphi }

\fc{\mm{rot}\vec{E}=0=-[\grad \times \grad \varphi]}  

\newpage

\subsection*{Работа электрического поля}

\fc{A=\underset{L}{\int}\vec{F}d\vec{l}=\underset{L}{\int}q\vec{E}d\vec{l}=q\left[\int_{\vec{r_1}}^{\vec{r_0}}\vec{E}d\vec{l}- \int_{\vec{r_2}}^{\vec{r_1}}\vec{E}d\vec{l}   \right]=q(\varphi_2-\varphi_2)=qU}

\subsection*{Потенциал точечного заряда}

\fc{\varphi(r) = \frac{q}{r}}

или в общем виде

\fc{\varphi(\vec{r}) = \underset{i}{\Sigma} \frac{q_i}{|\vec{r} - \vec{r}_i|}}

\section*{4. Уравнение Лапласа. Разделение переменных в уравнении Лапласа в
декартовой системе координат.}

\subsection*{Уравнение Лапласа}

В декартовой системе координат

\fc{\displaystyle {\frac {\partial ^{2}\varphi}{\partial x^{2}}}+{\frac {\partial ^{2}\varphi}{\partial y^{2}}}+{\frac {\partial ^{2}\varphi}{\partial z^{2}}}=0}

В сферической системе ($r,\theta,\alpha$) координат

\fc{{\displaystyle {1 \over r^{2}}{\partial  \over \partial r}\left(r^{2}{\partial \varphi \over \partial r}\right)+{1 \over r^{2}\sin \theta }{\partial  \over \partial \theta }\left(\sin \theta {\partial \varphi \over \partial \theta }\right)+{1 \over r^{2}\sin ^{2}\theta }{\partial ^{2}\varphi \over \partial \alpha ^{2}}=0}}

В цилиндрической ($r,\alpha,z$) координат

\fc{{\displaystyle {1 \over r}{\partial  \over \partial r}\left(r{\partial \varphi \over \partial r}\right)+{\partial ^{2}\varphi \over \partial z^{2}}+{1 \over r^{2}}{\partial ^{2}\varphi \over \partial \alpha ^{2}}=0}}

\subsection*{Разделение переменных в уравнении Лапласа в
декартовой системе координат}

Предположим, что в декартовых координатах переменные разделяются -это означает, что: 
\fc{\varphi(x,y,z)=X(x)\cdot Y(y)\cdot Z(z)}

\fc{\Delta \varphi=0 \Rightarrow X''YZ+XY''Z+XYZ''=0}

\newpage


\fc{\frac{X''}{X}+\frac{Y''}{Y}+\frac{Z''}{Z}=0 \Rightarrow Const_1+C_2+C_3=0}

\fc{\frac{X''}{X}=C \Rightarrow X''=CX}

\fc{(1)X(x)=
\left\{
\begin{aligned}
\text{при C} >0 ,Ae^{\sqrt{c}x} \\
\text{при C} <0 ,Ae^{\pm i\sqrt{c}x} \\
\text{при C} =0 ,Ax+B
\end{aligned}
\right.}

При $\rho \neq0$. Допустим, что 

\fc{\rho(x,y,z)=\rho \cdot X(x)\cdot Y(y)\cdot Z(z),\text{где X,Y,Z функции вида (1)}}

Тогда 

\fc{\varphi=A\cdot X(x)\cdot Y(y)\cdot Z(z)}

\fc{A(X''YZ+XY''Z+XYZ'')=-4\pi \rho_0 XYZ\Rightarrow \frac{X''}{X}+\frac{Y''}{Y}+\frac{Z''}{Z}=-\frac{4\pi \rho_0}{A}}

\fc{C_1+C_2+C_3=-\frac{4\pi \rho_0}{A}}

Итог

\fc{\rho = p _ { 1 } + p _ { 2 } , \Delta \varphi = - 4 \pi \rho , \varphi = \varphi _ { 1 } + \varphi _ { 2 } ;}

\fc{
\left\{
\begin{aligned}
\Delta \varphi_1=-4\pi \rho_1 \\
\Delta \varphi_2=-4\pi \rho_2  \
\end{aligned}
\right.}



\newpage


\section*{5. Уравнение Лапласа. Разделение переменных в уравнении Лапласа в
сферической системе координат.}

\subsection*{Уравнение Лапласа(повтор)}

\subsection*{Разделение переменных в уравнении Лапласа в
сферической системе координат}

Пусть $\varphi(r,\theta,\alpha)=R(r)\cdot Y(\theta)$

\fc{\Delta \varphi(r,\theta,\alpha)=0\Rightarrow \underset{=l(l+1)}{\frac{1}{R}\frac{d}{dr} \left( r^2 \frac{dR}{dr} \right)}+\underset{=-l(l+1)}{\frac{1}{Y \sin \theta}\frac{d}{d\theta} \left( \sin \theta \frac{dY}{d\theta} \right)}=0}

При $R(r)\varpropto r'$

или $R(r)\varpropto \frac{1}{r^{l+1}}$

\fc{\frac{1}{R}(r^2R')'=C} 

ищем решение в виде $R(r)\varpropto r^l$

\fc{\frac{1}{R}(r^2R')'=\underset{=-(l'+1)}{l}\cdot \underset{=(-l'-1+1)=(l'+1)l'}{(l+1)}} 

При этом $R(r)\varpropto \frac{1}{r^{l+1}}$ удолетвор. уравнению с той же С 

(замена $l'=-(l+1)$) 

\section*{6. Уравнение Лапласа. Разделение переменных в уравнении Лапласа в
цилиндрической системе координат.}

\subsection*{Уравнение Лапласа(повтор)}

\subsection*{Разделение переменных в уравнении Лапласа в
цилиндрической системе координат}

Пусть $\varphi(r,\alpha)=\varphi(r,\alpha)$. Кроме того $\varphi(r,\alpha)=R(z)Y(\alpha)$
(то есть переменные разделяются)

\fc{Y(\alpha)=e^{\pm im\alpha}}

\fc{\varphi(r,\alpha)=R(r)(\underset{i}{\Sigma}e^{ im\alpha}),\text{где m}\in Z}

Пусть внутри, рассматриваемой области нет зарядов $\Rightarrow \Delta \varphi=0 \Rightarrow$

\fc{\Rightarrow \frac{1}{r} \frac{\partial}{\partial r}\left(r \frac{\partial \varphi}{\partial r}\right)+\frac{1}{r^2} \frac{\partial^2 \varphi}{\partial \alpha^2}=0 \Rightarrow e^{i m \alpha} \cdot \frac{1}{r}\left(r R^{\prime}\right)^{\prime}+R \cdot \frac{1}{r^2}\left(-m^2 e^{i m \alpha}\right)=0 \Rightarrow \frac{r(rR')'}{R}=m^2}

Ищем решение в виде $R(r)\varpropto r^l$:
\fc{l^2=m^2 ,\text{т.е }l=\pm m \text{. Т.е } \varphi(r,\alpha)=\left(\frac{C_1}{r^m}+C_2r^m\right)e^{\pm im\alpha}}

\section*{7. Граничные условия для нормальной и тангенциальной компонент
электрического поля. Поверхностная плотность зарядов. Поле вблизи
поверхности металлов. Граничные условия для электрического поля,
выраженные через его скалярный потенциал.}

\subsection*{Граничные условия для нормальной и тангенциальной компонент
электрического поля}

\kr{Тангенциальная компонента}

\fc{\mm{rot}\vec{E}=0 \Rightarrow \oint \vec{E}d\vec{l}\leftarrow \text{интегральная форма}}

По теореме Стокса

\fc{0=\underset{(\forall )S}{\iint} \mm{rot}\vec{E}dS=\underset{(\forall )S}{\oint }\vec{E}d\vec{l}}

\imc[0.826\textwidth]{10.png} 

\newpage


\fc{\oint \vec{E}d\vec{l}=E_x|_1\cdot l-E_x|_2\cdot l\Rightarrow E_x|_1=E_x|_2}

или же 

\fc{E_\tau|_1=E_\tau|_2}

\kr{Нормальная компонента}

\fc{\mm{div}\vec{E}=4\pi \rho \Rightarrow \underset{(\forall)S}{\oiint}\vec{E}d\vec{S}=4\pi Q \Rightarrow}

\fc{\Rightarrow \underset{(\forall)V}{\iiint} \mm{div}\vec{E}dV=4\pi \underset{(\forall)V}{\iiint} \rho dV \Rightarrow \underset{(\forall)\delta S}{\oiint}\vec{E}d\vec{S}=4\pi Q }

\imc[0.7\textwidth]{11.png}

\fc{E_{1n}|\cdot S-E_{2n}|\cdot S=4\pi Q=4\pi \rho S}

или же 

\fc{E_{1n}| -E_{2n}| =4\pi \rho }

\subsection*{Поверхностная плотность зарядов(???)}

\fc{dq\overset{df}{=}\sigma dS}

\newpage


\subsection*{Поле вблизи поверхности металлов}

Надо доказать что поле вблизи металлов равно

\fc{\vec{E}=4\pi \sigma \vec{n}}

Рассмотрим тангенсальную и нормальную компоненту поля $\vec{E}$ на границе металла
 
\imc[0.7\textwidth]{12.png}

Если в проводнике имеется электрическое поле, то по нему течёт ток. Следовательно, для электростатических явлений электрическое поле внутри проводника $E_{1n}=0$ отсюда 

\fc{E_{2n}=4\pi \sigma}

Снаружи металла поле $E_{2\tau }=0$ и из граничных условий  

\fc{E_{1\tau}=0}

Итоговое поле равно 

\fc{\vec{E}_{2n}=4\pi \sigma \vec{n}}

Что и требовалось доказать.

\newpage


\subsection*{Граничные условия для электрического поля,
выраженные через его скалярный потенциал}

Можно рассмотреть две точки А и B с одной стороны поверхности и C,D с другой стороны. Найдем напряжение между парами этих  точек:

%Использовал пакет TikZ для пробы
\begin{center}
\begin{tikzpicture}
    % Ось X
    \draw[->] (-3,0) -- (3,0) node[right] {$x$};

    % Верхний вектор от A к B
    \node[above] at (-2,1) {$A$};
    \node[above] at (2,1) {$B$};
    \draw[->, thick] (-2,1) -- (2,1);

    % Нижний вектор от C к D
    \node[below] at (-2,-1) {$C$};
    \node[below] at (2,-1) {$D$};
    \draw[->, thick] (-2,-1) -- (2,-1);
\end{tikzpicture}
\end{center}

Из граничных условий, что $E_{\tau}-$непрерывно следует, что:

\fc{E_{\text{AB}|} = E_{\text{CD}}|} 

потенциал можно выразить через напряженность так:

\fc{{E}=-\mm{grad}\varphi}

отсюда получаем, что $\varphi_{\text{AB}}|=\varphi_{\text{CD}}|\Rightarrow \varphi|-\text{непрерывно}$

\section*{8. Проводники в электрическом поле. Теорема единственности.}

\subsection*{Проводники в электрическом поле}

Очень похоже (скорее всего есть одно и тоже) на вопрос: поле вблизи поверхности металлов, ну рассмотрим повторно?

\imc[0.4\textwidth]{13.png}

\newpage

Если в проводнике имеется электрическое поле, то по нему течёт ток. Следовательно, для электростатических явлений электрическое поле внутри проводника $E_{i}\equiv0$ отсюда плотность заряда: 

\fc{\rho_i=\frac{1}{4\pi}\mm{div}\vec{E_i}\equiv0}

В этой связи говорят, что проводник квазинейтрален. Таким образом,
заряды на проводнике могут размещаться только на его поверхности,
причем поверхностная плотность зарядов связана с полем $vec{E}$ вне про-водника через граничное условие для $E_n$.

Если пространство вне проводника свободно от зарядов, то здесь поле 
$\vec{E}=-\mm{grad}\varphi$ и $\varphi$ удовлетворяет уравнению Лапласа.

Из граничных условий мы получаем что:

\fc{\vec{E}_{n}=4\pi \sigma \text{ , } \vec{E}_{\tau}=0 .}

Заметим, что поле подходит к поверхности проводника по нормали, т.е. поверхность
проводника является эквипотенциалью. Это естественно, так как в проводнике потенциал постоянен из-за $\vec{E_i}=0$

\subsection*{Теорема единственности}

\kr{Условия теоремы:}

1)На каждом проводнике задан либо потенциал, либо заряд,

2)В $V$ нету зарядов;

$\Rightarrow \exists$ единственное решение уравнения Пуассона вида:

\fc{\vec{E}=-\grad\varphi }

\kr{Доказательсвто}

Пусть $\vec{E_1}=-\grad\varphi_1$ и $\vec{E_2}=-\grad\varphi_2$. Достаточно доказать, что:

\fc{\underset{V}{\iiint}|\vec{E_2}(\vec{r})-\vec{E_1}(\vec{r})|^2dV=0}

\fc{\vec{E}:=\vec{E_2}-\vec{E_2} \text{ ; } \varphi:=\varphi_2-\varphi_1 \text{ ; } \vec{E} =-\grad\varphi_2+\grad\varphi_1=-\grad\varphi   }

Всюду в $V$ $\Delta\varphi_1=0$ и $\Delta\varphi_2=0\Rightarrow \Delta\varphi=0$

Рассмотрим выражение: $\grad(\varphi\grad\varphi)=(\grad\varphi)^2+\underset{\rightarrow0}{\varphi\grad\varphi}=(\grad\varphi)^2$

\fc{{\iiint}|\vec{E_2}(\vec{r})-\vec{E_1}(\vec{r})|^2dV={\iiint}|\vec{E}|^2dV=\iiint(\grad\varphi)^2dV=\iiint\grad(\varphi\grad\varphi)dV=}

\fc{=\underset{V}{\iiint}\mm{div}(\varphi\grad\varphi)dV= \underset{\rightarrow0(\propto \frac{1}{r})}{\underset{S_\infty}{\oiint}\varphi\grad\varphi d\vec{S}}-\underset{i}{\Sigma}\underset{S_i}{\oiint}\varphi\grad\varphi d\vec{S}=\underset{i}{\Sigma}\varphi_i \underset{S_i}{\oiint}(-\grad\varphi) d\vec{S}=}

\fc{\underset{i}{\Sigma}(\varphi_{2i}-\varphi_{1i})\underset{S_i}{\oiint}(\vec{E_2}-\vec{E_1})d\vec{S}=\underset{i}{\Sigma}(\varphi_{2i}-\varphi_{1i}) \left[\underset{S_i}{\oiint}\vec{E_2}d\vec{S}-\underset{S_i}{\oiint}\vec{E_1}d\vec{S}\right]=}

\fc{=4\pi \underset{i}{\Sigma}\underset{(1)}{(\varphi_{2i}-\varphi_{1i})}\underset{(2)}{(q_{2i}-q_{1i})}=0}

По условию теоремы либо (1) = 0, либо (2)=0

\kr{Доказано}

\section*{9. Метод изображения для решения задач электростатики на примере плоской и сферической границ раздела проводника и непроводящего пространства.}

\kr{Плоская граница}

Точечный заряд  $q$ , находящийся на расстоянии  $h$  от проводящего полупространства. Определить поле в свободном полупространстве и на этой основе — плотность зарядов, индуцированных зарядом  $q$  на поверхности проводника.

\imc[0.6\textwidth]{14.png}

В проводящем полупространстве поле равно нулю, постоянный потенциал можно принять за ноль, будем искать поле только в области $z>0$ с выкинутой точкой. Искомое поле удовлетворяет уравнению Лапласа:

\fc{\Delta \varphi=0}

и граничным условиям

\fc{\varphi|_{z=0}=0 \text{ , } \underset{S_\varepsilon}{\oint}E_n dS=4 \pi Q}

\newpage

где $S_\varepsilon $ сфера малого радиуса с центром в точке
расположения заряда $q$

В проводящем полупространстве будет наводится заряд $q'=-q$. Тогда потенциал и электрическое поле, созданные зарядом $q$ фиктивным зарядом $q'$
,в правом полупространстве создают искомое поле:

\fc{\varphi=\frac{q}{r}-\frac{q}{r_1}}

Действительно, эта функция удовлетворяет уравнению Лапласа в
области $z>0$ как потенциал двух точечных зарядов, лежащих вне области. Во-вторых, $\varphi |_{z=0}=0$, так как для точек плоскости $r$ и $r_1$ равны.
В-третьих, поле, созданное зарядом $q'$, через поверхность $S\varepsilon$ создает
поток, равный нулю (по теореме Гаусса), а поле от точечного заряда
$q$ обеспечивает выполнение соответствующего граничного условия.

\fc{\varphi(\vec{r}) =
\left\{
\begin{aligned}
\frac{q}{|\vec{r}|}-\frac{q}{|\vec{r_1}|}\text{ , }z\geq0 \\
 0	\text{, z<0}	\\
\end{aligned}
\right.}

и 

\fc{\vec{E}(\vec{r}) =
\left\{
\begin{aligned}
\frac{q}{|\vec{r}|^2}\frac{\vec{r}}{|\vec{r}|}-\frac{q}{|\vec{r_1}|^2}\frac{\vec{r_1}}{|\vec{r_1}|}\text{ , }z\geq0 \\
 0	\text{, z<0}	\\
\end{aligned}
\right.}

Таким образом, задача решена.



\kr{Для сферической границы}

Заряд $q$ на расстоянии $l+x$ от центра шара, а потенциал шара принят равным нулю. 

\imc[0.55\textwidth]{15.png}

Искомый потенциал в
произвольной точке P вне шара в этом случае:

\fc{\varphi(P)=\frac{q}{r}+\frac{q'}{r'}}

где $q'=-q\frac{a}{l}$

\newpage

 Решение удовлетворяет уравнению Лапласа в своей области определения, имеет нужную особенность вблизи точечного заряда q и удовлетворяет граничным условиям ($r_0/r_0'=l/a$), обращаясь в нуль.
 
Рассмотрим второй вариант — точечный заряд $q$ рядом с шаром,
несущим заряд $Q$ (при этом постоянный потенциал шара не определен). В этом случае к существующей системе зарядов $q,q'$ необходимо добавить фиктивный заряд, расположенный в центре шара:

\fc{q''=Q-q'=Q-q\frac{a}{l}}

тогда 

\fc{\varphi(P)=\frac{q}{r}+\frac{q'}{r'}+\frac{q''}{r_*}}

потенциал шара при этом:

\fc{\varphi(P)=\varphi|_S =\frac{q}{r_0}+\frac{q'}{r'_0}+\frac{q''}{a}\Rightarrow \varphi|_0=\frac{q''}{a}=\frac{Q}{a}+\frac{q}{l}}

Таким образом, задача решена.

\section*{10. Электрический диполь. Потенциал и поле диполя.}

\subsection*{Электрический диполь}

Пусть система зарядов занимает ограниченную область пространства с характерным размером $a$, причем начало координат находится внутри этой области.

\imc[0.5\textwidth]{16.png}

Распишем потенциал точечных зарядов:

\fc{\varphi(\vec{r})=\underset{i}{\Sigma}\frac{q_i}{|\vec{r}-\vec{r_i'}|}=:\Sigma \frac{q}{|\vec{r}-\vec{r'}|}}

\newpage

Используем разложение:

\fc{\frac{1}{|\vec{r}-\vec{r'}|}=\frac{1}{r}+(-\vec{r'})\gradd\frac{1}{r}+...=\frac{1}{r}+(-r')(-\frac{1}{r^2}\cdot\frac{\vec{r}}{r})+...=\frac{1}{r}+\frac{\vec{r}\vec{r'}}{r^3}}

получаем

\fc{\varphi=\Sigma q\frac{1}{r}+\Sigma q\vec{r'}\frac{\vec{r}}{r^3}+...=\frac{Q}{r}+\frac{\vec{d}\vec{r}}{r^3}+...}

где

\kr{Дипольный момент-} $\vec{d}:=\underset{i}{\Sigma}q_i\vec{r_i'}$

Полный заряд системы-$Q=\underset{i}{\Sigma}q_i$

Дипольный член в сферических координатах( $\vec{e_z}\uparrow\uparrow\vec{d}$ ):

\fc{\varphi(r,\theta)=\frac{d}{r^2}\cos \theta}

\subsection*{Потенциал и поле диполя}

Из прошлого пункта:

\fc{\varphi=\frac{Q}{r}+\frac{\vec{d}\vec{r}}{r^3}}

Найдем поле диполя:

\fc{\vec{E}=-\grad \varphi=-\grad \left((\vec{d}\vec{r})\frac{1}{r^3})\right)=-\grad \left(\overset{\downarrow}{(\vec{d}\vec{r})}\frac{1}{r^3}\right)-\grad \left((\vec{d}\vec{r})\overset{\downarrow}{\frac{1}{r^3}}\right)=}

\fc{=-\frac{1}{r^3}\grad (\vec{d}\vec{r})-(\vec{d}\vec{r})\grad\frac{1}{r^3}=-\frac{\vec{d}}{r^3}+3\frac{(\vec{d}\vec{r})}{r^4}\grad \vec{r}=-\frac{\vec{d}}{r^3}+3\frac{(\vec{d}\vec{r})\vec{r}}{r^5}}

Итог, \kr{поле диполя:}

\fc{\vec{E}=-\frac{\vec{d}}{r^3}+3\frac{(\vec{d}\vec{r})\vec{r}}{r^5}}

\newpage

\section*{11. Сила и момент сил, действующие на диполь в слабонеоднородном
электрическом поле.}

\kr{Момент сил:}

Рассмотрим случай двух зарядов:

\imc[0.45\textwidth]{17.png}


\fc{\vec{F} = q\vec{E} \text{ , } \vec{M} = [\vec{r} \times \vec{F}]=[\vec{r} \times q\vec{E}]=[\vec{d} \times \vec{E}]}

Обобщим на случай нескольких зарядов:

\imc[0.45\textwidth]{18.png}

\fc{\vec{M}=\underset{i}{\Sigma}[\vec{r_i} \times \vec{F_i}]=\underset{i}{\Sigma}[\vec{r_i} \times q_i\vec{E}]=\underset{i}{\Sigma}[q_i\vec{r_i} \times \vec{E}]=[(\underset{i}{\Sigma}q_i\vec{r_i}) \times \vec{E}]=[\vec{d} \times \vec{E}]}

\newpage

\kr{Сила:}

Рассмотрим случай двух зарядов:

\imc[0.45\textwidth]{19.png}

В однородном поле $F=0$, если полный заряд равен нулю:

\fc{\vec{F}=\underset{i}{\Sigma}q_i\vec{E}=\underset{\rightarrow0}{(\underset{i}{\Sigma}q_i)}\vec{E}=0}

\fc{\vec{F}=q\vec{E}(d\vec{r}+d\vec{r})-q\vec{E}(\vec{r})=q(\vec{E}(d\vec{r}+d\vec{r})-\vec{E}(\vec{r}))=q(d\vec{r}\gradd)\vec{E}=(\vec{d}\gradd)\vec{E}}
с учетом , что $\mm{rot}\vec{E}=0$:

\fc{0=[\grad \times \vec{E}]\Rightarrow 0=[\vec{d}\times[\grad\times\vec{E}]]\underset{bac-cab}{=}\grad(\vec{d}\vec{E})-\overset{\downarrow}{\vec{E}}(\vec{d}\grad)\Rightarrow \grad(\vec{d}\vec{E})=(\vec{d}\grad)\vec{E}}

Получаем нашу силу:

\fc{\vec{F}=\gradd(\vec{d}\vec{E})}

Можно ввести потенциальную функцию по общему правилу:

\fc{\vec{F}=-\grad U}

Тогда 

\fc{U=-\vec{d}\vec{E}}

\newpage

Обобщим на случай нескольких зарядов:

\imc[0.45\textwidth]{20.png}

Предполагается, что система мала по сравнению с масштабами изменения электрического поля:

\fc{\vec{F}=\underset{i}{\Sigma}q_i\vec{E_i}(\vec{r}+\vec{r_i})}

с учетом, что $\underset{i}{\Sigma}q_i=0$, получим:

\fc{\vec{F}=\underset{i}{\Sigma}q_i(\vec{E}(\vec{r}+\vec{r_i})-\vec{E}(\vec{r}))=\underset{i}{\Sigma}q_i(\vec{r_i}\grad)\vec{E}=\underset{i}{\Sigma}(q_i\vec{r_i}\grad)\vec{E}=((\underset{i}{\Sigma}q_i)\vec{r_i}\grad)\vec{E}=(\vec{d}\grad)\vec{E}}

Получим нашу силу:

\fc{\vec{F}=\grad(\vec{d}\vec{E})}

и 

\fc{U=-\vec{d}\vec{E}}

В случае упругого диполя:

\fc{\vec{d}\overset{df}{=}\alpha\vec{E}}

тогда запишем нашу силу:

\fc{\vec{F}=\grad(\vec{d}\cdot\overset{\downarrow}{\vec{E}})=\grad(\alpha\vec{E}\cdot\overset{\downarrow}{\vec{E}})=\frac{1}{2}\left[\grad(\overset{\downarrow}{\alpha\vec{E}}\cdot \vec{E})+\grad(\alpha\vec{E}\cdot\overset{\downarrow}{\vec{E}})\right]=}
\fc{=\frac{1}{2}\grad(\alpha\overset{\downarrow}{\vec{E}}\cdot \overset{\downarrow}{\vec{E}})=\frac{1}{2}\grad(\overset{\downarrow}{\vec{d}}\cdot\overset{\downarrow}{\vec{E}})=\vec{F}}

с учетом $\vec{F}=-\grad U$, получаем $U=-\frac{1}{2}\vec{d}\vec{E}$

\newpage

\section*{12. Электрический квадрупольный момент. Тензор квадрупольного момента для аксиально-симметричной системы зарядов.}

\subsection*{Электрический квадрупольный момент}

\imc[0.45\textwidth]{21.png}

Точное решение:

\fc{\varphi=\underset{i}{\Sigma}\frac{q_i}{|\vec{r}-\vec{r_i'}|}=:\Sigma \frac{q}{|\vec{r}-\vec{r'}|}}

Нужно разложить $\frac{1}{\vec{r}-\vec{r'}}$. Перейдем к тензорной записи:

\fc{\vec{r}(x,y,z)=:(x_1,x_2,x_3)\rightarrow x_{\alpha}, \text{ анолгично }\vec{r'}-x_{\alpha}'}
Индексы $\alpha,\beta \in [1,2,3]$

По сути раскладываю функцию:

\fc{\frac{1}{r}=\frac{1}{\sqrt{x_1^2+x_2^2+x_3^2}}}

\fc{\frac{1}{|\vec{r}-\vec{r'}|}=\underset{\underset{\text{монополь}}{l=0}}{\frac{1}{r}}+\underset{\underset{\text{диполь}}{l=1}}{(-x_{\alpha}')\frac{\partial}{\partial x_{\alpha}}\frac{1}{r}}+\underset{\underset{\text{квадруполь}}{l=2}}{\frac{1}{2}(-x_{\alpha}')(-x_{\beta}')\frac{\partial^2}{\partial x_{\alpha}\partial x_{\beta}}\frac{1}{r}}+...}

\fc{\frac{\partial^2}{\partial x_{\alpha}\partial x_{\beta}}\frac{1}{r}-?}

Найдем:

\fc{\frac{\partial}{\partial x_1}\cdot \frac{1}{\sqrt{x_1^2+x_2^2+x_3^2}}=-\frac{1}{2r^3}2x_1=-\frac{x_1}{r^3}}

\newpage

\fc{\frac{\partial^2}{\partial x_1 \partial x_2}\frac{1}{r}=\frac{\partial}{\partial x_1}\left(-\frac{x_1}{r^3}\right)=-x_1 \left(-\frac{1}{r^4}\frac{x_2}{r} \right)=\frac{3 x_1 x_2}{r^5}}

\fc{\frac{\partial^2}{\partial x_1 \partial x_2}\frac{1}{r}=-\frac{\partial}{\partial x_1}\left(-\frac{x_1}{r^3}\right)=-\frac{1}{r^3}\cdot 1+\frac{3x_1x_2}{r^5}}

\fc{\Downarrow}

\fc{\frac{\partial^2}{\partial x_{\alpha}\partial x_{\beta}}=\frac{-\delta_{\alpha\beta}r^2+x_{\alpha}x_{\beta}}{r^5}}

Таким образом квадрупольный член имеет вид:

\fc{\varphi=\Sigma q\frac{1}{2}x_{\alpha}'x_{\beta}'\left( \frac{-\delta_{\alpha\beta}r^2+x_{\alpha}x_{\beta}}{r^5}\right)}

\fc{Q_{\alpha\beta}:=\Sigma \frac{1}{2}qx_{\alpha}x_{\beta}}

тогда 

\fc{\varphi=Q_{\alpha\beta}\frac{-\delta_{\alpha\beta}r^2+x_{\alpha}x_{\beta}}{r^5}}

\fc{Tr\left(\frac{-\delta_{\alpha\beta}r^2+x_{\alpha}x_{\beta}}{r^5}\right)=\frac{-\delta_{\alpha\beta}r^2+x_{\alpha}x_{\beta}}{r^5}=}

\fc{=\frac{-(\delta_{11}+\delta_{22}+\delta_{33})r^2+3(x_1x_1+x_2x_2+x_3x_3)}{r^5}=\frac{-3r^2+3r^2}{r^5}=0}

Хотим:

\fc{\varphi=D_{\alpha\beta}\frac{x_{\alpha}x_{\beta}}{r^5}\text{ Как найти } D_{\alpha\beta} ?}

\fc{D_{\alpha\beta}:3Q_{\alpha\beta}-?\cdot\delta_{\alpha\beta}r'^2 \text{ подберем ? так, чтобы } Tr(D_{\alpha\beta})=0, \text{так как }\rightarrow}

\fc{\rightarrow \text{при этом }D_{\alpha\beta}\delta_{\alpha\beta}r^2=0(=D_{\alpha\beta}=0)}

\fc{D_{\alpha\beta}=3x_{\alpha}x_{\beta}-\delta_{\alpha\beta}r'^2 \text{ Действительно } Tr(D_{\alpha\beta})=D_{\alpha\alpha}=3r'^2-3r'^2=0}

Тогда 

\fc{\varphi=\Sigma q(3x_{\alpha}'x_{\beta}'-\delta_{\alpha\beta}r'^2)\cdot\frac{x_{\alpha}x_{\beta}}{2r^5}}

Таким образом $D_{\alpha\beta}=\Sigma q(3x_{\alpha}'x_{\beta}'-\delta_{\alpha\beta}r'^2)$

\newpage

\subsection*{Тензор квадрупольного момента для аксиально-симметричной системы зарядов}

\imc[0.4\textwidth]{22.png}

Вопрос:

\fc{D_{xy}'-?}
\fc{\rotatebox{90}{=}}
\fc{D_{12}'-?}
\fc{-----------------------}
\fc{x_1'x_2'=x'y'=(-y)x=-xy|\text{ или }x_1'x_2'=-x_1x_2\Rightarrow}
\fc{\Rightarrow D_{12}'=-D_{12}}
Должно быть $D_{12}'=D_{12}$ из-за симметрии, поэтому имеем:

\fc{D_{12}=0}

\fc{\overset{\wedge}{D}=\begin{pmatrix}
-\frac{1}{2}D & 0 & 0 \\
0 & -\frac{1}{2}D & 0 \\
0 & 0 & -\frac{1}{2}D
\end{pmatrix}}

Что?

\fc{\varphi_2=D_{\alpha\beta}\frac{x_{\alpha}x_{\beta}}{2r^5}=\frac{1}{2r^5}(x_1,x_2,x_3)
\begin{pmatrix}
-\frac{1}{2}D & 0 & 0 \\
0 & -\frac{1}{2}D & 0 \\
0 & 0 & D
\end{pmatrix}
\begin{pmatrix}
x_1 \\
x_2 \\
x_3
\end{pmatrix}=
}

\fc{=\frac{D}{2r^5}(x_1,x_2,x_3)
\begin{pmatrix}
-x_1/2 \\
-x_2/2 \\
x_3
\end{pmatrix}=
\frac{D}{2r^5}\left(-\frac{x_1^2+x_2^2}{2}+x_3^2 \right)
=\frac{D}{2r^5}\left(-\frac{x^2+y^2}{2}+z^2 \right)=
}

\fc{=\frac{D}{2r^5}\left(-r^2\frac{\sin^2\theta}{2}+r^2\cos^2\theta \right)=\frac{D}{2r^3}\cdot\frac{3\cos^2\theta-1}{2}}
где последний член это полином Лежанра $P(\cos \theta)$

\newpage

Вкратце о аксиально-симметричном тензоре:

\fc{D_{\alpha\beta}=\begin{pmatrix}
\frac{1}{2}D_{zz} & 0 & 0 \\
0 & \frac{1}{2}D_{zz} & 0 \\
0 & 0 & D_{zz}
\end{pmatrix}}

\fc{1) D_{xx}+D_{yy}+D_{zz}=0}
\fc{2) D_{xy}'=-D_{yx}=-D_{xy}(\text{ свойство тензора при повороте на }90^{\circ})}
\fc{D_{xy}'=D_{xy} (\text{ из симметрии })}
\fc{\Downarrow}
\fc{D_{xy}=0}
\fc{3) D_{xz}'=-D_{xz} (\text{ свойство тензора при повороте на }180^{\circ})}
\fc{D_{xz}'=D_{xz}}
\fc{\Downarrow}
\fc{D_{xz}=0}



\section*{13. Энергия электрического поля. Плотность энергии электрического поля.}

В обьеме $V : \Delta\varphi=0$
и граничные условия $\underset{S_i}{\oiint}(-\gradd\varphi) d\vec{S}=4\pi q_i \Rightarrow$ 

$\Rightarrow$задача линейна.

\imc[0.6\textwidth]{24.png}

\fc{\delta A=\varphi_idq_i 
\bigg| \;
\begin{array}{rl}
\tilde{\varphi}_i &= \alpha\varphi_i \\
\tilde{q}_i &= \alpha q_i \text{ , } \tilde{q_i}=q_id\alpha(0\leqslant\alpha\leqslant1)
\end{array}
}

\fc{\Delta A=\int \underset{i}{\Sigma}\tilde{\varphi_i}d\tilde{q_i}=\int \underset{i}{\Sigma}\alpha\varphi_id(\alpha q_i)=(\int_0^1\alpha d\alpha )\underset{i}{\Sigma}\varphi_i q_i=\frac{1}{2}\underset{i}{\Sigma}\varphi_i q_i=\frac{1}{2}\varphi_i q_i}

\fc{\underset{V}{\iiint}\frac{E^2}{8\pi}dV=\underset{V}{\iiint}\frac{(-\grad\varphi)^2}{8\pi}dV=[\grad(\varphi\grad\varphi)=(\grad\varphi)^2+\varphi\grad\varphi=(\grad\varphi)^2]=}

\fc{=\underset{V}{\iiint}\frac{1}{8\pi}\grad(\varphi\grad\varphi)dV=\frac{1}{8\pi} \bigg(\underset{i}{\Sigma}\underset{S_i}{\oiint}\varphi(-\grad\varphi)d\vec{S}+\underset{S_{\infty}}{\oiint}\underset{\rightarrow0}{\varphi(\grad\varphi)d\vec{S}}\bigg)=}
\fc{=\frac{1}{8\pi}\underset{i}{\Sigma}\varphi_{i}\underset{S_i}{\oiint}\vec{E}d\vec{S}=[\underset{S_i}{\oiint}\vec{E}d\vec{S}=4\pi q_i]=\frac{1}{2}\varphi_i q_i}

\kr{Плотность энергии: }$w=\frac{E^2}{8\pi}$, энергия $W=\iiint wdV$.

\section*{14. Электрическая ёмкость. Матрица емкостных коэффициентов, её симметричность.}

\subsection*{Электрическая ёмкость}

Конденсатор из двух проводников:

\imc[0.4\textwidth]{27.png}

\fc{C=\frac{|q|}{\varphi_1-\varphi_2}}

Уединенный конденсатор:

\imc[0.2\textwidth]{28.png}

\fc{C=\frac{q}{\varphi_1-\varphi_2}}

Энергия кондесатора:

\fc{W=\frac{q^2}{2C}=\frac{CU^2}{2}=\frac{1}{2} \Sigma q_i\varphi_i  }

\newpage

\subsection*{Матрица емкостных коэффициентов, её симметричность}

\imc[0.4\textwidth]{26.png}

\fc{\begin{pmatrix}
\varphi_1 \\
\varphi_2 \\
\vdots \\
\varphi_i \\
\vdots
\end{pmatrix}
=
\renewcommand{\arraystretch}{1.8} % Увеличивает вертикальные интервалы в матрице
\setlength{\arraycolsep}{1.2em}   % Увеличивает горизонтальные интервалы между столбцами
\begin{pmatrix}
   &   &   \\
   & \overset{\wedge}{S} &   \\
   &   &  
\end{pmatrix}
\quad
\renewcommand{\arraystretch}{1.} 
\setlength{\arraycolsep}{1.2em}   
\begin{pmatrix}
q_1 \\
q_2 \\
\vdots \\
q_i \\
\vdots
\end{pmatrix}
\quad
}

или в другом виде:

\fc{\varphi_{i}=S_{ij}q_{j}}

где $S_{ij}$-\kr{матрица потенциальных коэффициентов,}

\fc{q_{i}=C_{ij}\varphi_j}

где $С_{ij}$-\kr{матрица емкостных коэффициентов,}$S_{ij}^{-1}=C_{ij}$-симметричны.

Свойства матриц:

\fc{dW=\underset{i}{\Sigma}\varphi_i dq_i=\varphi_i dq_i}
и
\fc{W=\frac{1}{2}\varphi_i q_i=dW=\frac{1}{2}\varphi_i dq_i+\frac{1}{2}q_id\varphi_i}
\fc{\Downarrow}

\fc{\varphi_idq_i=q_id\varphi_i}

где $\varphi_i=S_{ij}q_j$

\fc{0=S_{i j} q_j d q_i-q_i S_{i j} d q_i=S_{i j} q_j  d q_i-q_j S_{j i} d q_i=(S_{i j}-S_{ji})q_idq_j}
получаем:

\fc{S_{ij}=S_{ji}}
\fc{\Downarrow}
\fc{C_{ij}=C_{ji}}

это справедливо $\forall q_i$ и $\forall dq_i$.

\newpage

\section*{15. Диэлектрики. Связанный заряд. Вектор поляризации. Электрическое поле
в диэлектрике. Вектор индукции. Диэлектрическая проницаемость.}

\subsection*{Диэлектрики}

\kr{Неполярный диэлеткрик}(к примеру $H_2,O_2$):

\imc[0.3\textwidth]{29.png}

\fc{\vec{d_i}=0}

Под действием поля $\vec{E}$ происходит смещение электронного облака
 $\text{ и}<\vec{d_i}\neq0>$

\kr{Ионный диэлектрик:}

\imc[0.4\textwidth]{30.png}

\fc{\vec{d_i}=0}
\fc{\text{Eсли }\vec{E}=0\Rightarrow <\vec{d_i}>=0}
\fc{\text{Eсли }\vec{E}\neq0\Rightarrow U=-\vec{d}\vec{E}\text{, }<\vec{d}>\neq0}

\newpage

\kr{Полярный диэлектрик:}

\imc[0.25\textwidth]{31.png}

\fc{\vec{d_i}\neq0}
\fc{\text{Eсли }\vec{E}=0\Rightarrow <\vec{d}>=0\text{, }U=-\vec{d}\vec{E}}

\subsection*{Связанный заряд и Вектор поляризации}

\imc[0.4\textwidth]{32.png}

\fc{\iiint<\rho_c>dV=0, \rho_c\text{-связанные заряды }}

\fc{<\rho_c>=\frac{1}{\Delta V}\underset{V}{\iiint}\rho_c(\vec{r}+\vec{\xi})d\vec{\xi}\text{, }\Delta V\sim 10^{-6}\text{см}^3}

Определение\kr{ вектора поляризации:}

\fc{<\rho_c>=:-\mm{div}\vec{P}}

Вне тела $\vec{P}=0$

\newpage

\imc[0.3\textwidth]{33.png}

По формуле Остроградского-Гаусса (1):

\fc{\iiint<\rho_c>dV=-\iiint\mm{div}\vec{D}dV\overset{(1)}{=}-\oiint\vec{P}d\vec{S}\Rightarrow}
\fc{\Rightarrow P_n=-\sigma_c}

Связь вектора поляризации и дипольного момента:

\fc{\vec{d}=\iiint <\rho_c>\vec{r}dV=-\iiint\vec{r}(\grad\vec{P})dV=-\iiint\grad\underset{\rightarrow0}{(\vec{P}\vec{r})}dV+\iiint(\vec{r}\grad)\vec{P}dV=}
\fc{\grad(\vec{P}\vec{r})=\vec{r}(\grad\vec{P})+(\vec{r}\grad)\vec{P}\text{ , }(\vec{r}\grad)\vec{P}=\bigg(x\frac{\partial}{\partial x}+y\frac{\partial}{\partial y}+z\frac{\partial}{\partial z}\bigg)(P_1,P_2,P_3)=\vec{P}}
\fc{=\iiint\vec{P}dV}
\fc{\Rightarrow\vec{P}=n<\vec{d}>}

Вектор поляризации $\vec{P}$ равен дипольному моменту единицы объема поляризованного диэлектрика.

\fc{\vec{P}=\chi\vec{E}, \chi \text{-поляризуемость.}}

\subsection*{Электрическое поле в диэлектрике и Вектор индукции и Диэлектрическая проницаемость}

\fc{
\begin{cases}
\mm{div}<\vec{E}>=4\pi(\rho+<\rho_c>) \qquad\qquad  \mm{div}(\vec{E}+4\pi\vec{P})=4\pi\rho\Rightarrow \mm{div}\vec{D}=4\pi\rho\\
\mm{rot}\vec{E}=0 \qquad\qquad\qquad\qquad\qquad\quad\quad \vec{D}:=\vec{E}+4\pi\vec{P}\text{(нет физического смысла)}
\end{cases}}

$<\vec{E}>:=\vec{E}$-напряженность электрического поля,

$\vec{D}$-\kr{вектор индукции электрического тока}.

По теореме Гаусса:

\fc{\oiint\vec{D}d\vec{S}=4\pi Q }

\fc{\underset{\text{Г}}{\oint}\vec{E}d\vec{l}=0\Rightarrow\vec{E}=-\grad\varphi\text{, }\varphi=-\int\vec{E}d\vec{l}}
\fc{\vec{D}=\vec{E}+4\pi\chi\vec{E}=(1+4\pi\chi)\vec{E}=\varepsilon\vec{E}}

$\varepsilon$\kr{-диэлектрическая проницаемость}($\varepsilon\geq1$)

\fc{\vec{P}=\frac{\vec{D}-\vec{E}}{4\pi}}

\fc{\rho_c=-\grad\bigg(\frac{\vec{D}-\vec{E}}{4\pi}\bigg)=-\grad\bigg(\frac{\varepsilon-1}{4\pi}\vec{E}\bigg)=\frac{1-\varepsilon}{4\pi}\grad\vec{E}-\vec{E}\frac{\grad\varepsilon}{4\pi}}

\kr{Итого имеем:}

\fc{
\begin{cases}
\mm{div}\vec{D}=4\pi\rho \qquad\qquad \oiint\vec{D}d\vec{S}=4\pi Q \\
\mm{rot}\vec{E}=0 \qquad\qquad\quad \oint\vec{E}d\vec{S}=0\\
\vec{D}=\varepsilon\vec{E}
\end{cases}}

\section*{16. Уравнения электрического поля в диэлектрике. Граничные условия.}

\subsection*{Уравнения электрического поля в диэлектрике}

\fc{
\text{Усреднение:}
\begin{cases}
\text{div}\, \vec{E} = 4\pi (\langle \rho_c \rangle + \rho) & \quad \langle \rho_c \rangle = - \text{div}\, \vec{P} \\
\text{rot}\, \vec{E} = 0 & \quad \vec{P} = \chi \langle \vec{E} \rangle
\end{cases}
}
\fc{
\text{Обозначения:}
\begin{array}{rl}
<\vec{E}>=:\vec{E} 
 \begin{array}{rl}
\end{array}   \\
\vec{E}+4\pi\vec{P}=:\vec{D}
\end{array}
}
\fc{
\begin{cases}
\mm{div}\vec{D}=4\pi\rho \\
\mm{rot}\vec{E}=0 \qquad \vec{E}=-\grad\varphi	
\end{cases}
}

\fc{\text{Вектор индукции: }\vec{D}=\vec{E}+4\pi\vec{P}=(1+4\pi\chi)\vec{E}=\varepsilon\vec{E}}
\newpage

\subsection*{Граничные условия}

\kr{Тангенсальная компонента:}

\imc[0.4\textwidth]{34.png}

\fc{\begin{cases}
\mm{rot}\vec{E}=0 \\
\underset{L}{\oint}\vec{E}d\vec{l}=0 \qquad \Rightarrow E_{1\tau}|=E_{2\tau}|
\end{cases}
}

То есть тангенсальная компонента вектора напряжённости электрического поля на границе непрерывна, а так же:

\fc{\varphi_1|=\varphi_2|}

\kr{Нормальная компонента:}

\imc[0.55\textwidth]{35.png}

\fc{\begin{cases}
\mm{div}\vec{D}=4\pi\rho \\
\underset{S}{\oiint}\vec{D}d\vec{S}=4\pi Q \qquad \Rightarrow D_{1n}|-D_{2n}|=4\pi\sigma\text{ или } \varepsilon_1E_{1n}-\varepsilon_2E_{2n}=4\pi\sigma
\end{cases}
}

То есть нормальная компонента вектора
индукции электрического поля
терпит разрыв.

\newpage

\section*{17. Оценка диэлектрической проницаемости полярного диэлектрика (газа).}


\end{document}