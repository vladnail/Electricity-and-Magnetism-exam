

\section*{18. Локальное поле в диэлектрике (поле Лоренца). Формула Клаузиуса –
Моссотти.}

\subsection*{Локальное поле в диэлектрике (поле Лоренца)}

\imc[0.8\textwidth]{37.png} 



\imc[0.6\textwidth]{38.png} 

При усреднении по углам:

\fc{<r^2>=<x^2+y^2+z^2>\underset{\text{изотропность }}{=}3<z^2>}
\fc{<\vec{E}>=\bigg<-\frac{\vec{d}}{r^3}+\frac{3(d\vec{d}\vec{r})\vec{r}}{r^5}\bigg>=\bigg< \frac{d}{r^3}(-r^2+3z^2)\vec{e_z}\bigg>=0}
\newpage


\fc{\vec{E_{\text{л}}}=\vec{E}-\vec{E_{\text{ш}}}\text{(по принципу суперпозиции)}}


\fc{E_{\tau}|_A-\text{непр.}:-\frac{\vec{d}}{r^3}+\frac{3\overset{\rightarrow0}{(\vec{d}\vec{r})\vec{r}}}{r^5}}
\fc{
\begin{cases}
	
	\vec{E_{\text{л}}}=\vec{E}-\vec{E_{\text{ш}}}=\vec{E}+\frac{\vec{d}}{r^3}\\
	\vec{d}  =\frac{4}{3}\pi r^3\vec{P}
	\end{cases}}
	
\fc{\Rightarrow \vec{E_{\text{л}}}=\vec{E}+\frac{4}{3}\pi\vec{P} }

\subsection*{Формула Клаузиуса-Моссотти}

\fc{
\begin{array}{l|ll}
\vec{d} = l \vec{E_l}  & \Rightarrow (\varepsilon-1)\vec{E}=4\pi\vec{P}\Rightarrow\vec{E}=\frac{4\pi}{\varepsilon-1}\vec{P} \\[10pt]
\vec{P} = n \vec{d} & \vec{E_{\text{л}}}=\frac{4\pi}{\varepsilon-1}\vec{P}+\frac{4}{3}\pi\vec{P}=\frac{4\pi}{3}\bigg(\frac{\varepsilon+2}{\varepsilon-1}\bigg)\vec{P}=\frac{4\pi}{3}\bigg(\frac{\varepsilon+2}{\varepsilon-1}\bigg)n\alpha\vec{E_\text{л}}\Rightarrow\\[10pt]
\vec{E_{\text{л}}} = \vec{E} + \frac{4}{3} \pi \vec{P} & \Rightarrow \kr{формула Клазиуса-Моссоти:} \\[10pt]
\varepsilon \vec{E} = \vec{E} + 4 \pi \vec{P} &\qquad\qquad\qquad \boxed{\frac{\varepsilon+2}{\varepsilon-1}=\frac{4\pi}{3}n\alpha}\\
\end{array}
}