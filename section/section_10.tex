\section*{10. Электрический диполь. Потенциал и поле диполя.}

\subsection*{Электрический диполь}

Пусть система зарядов занимает ограниченную область пространства с характерным размером $a$, причем начало координат находится внутри этой области.

\imc[0.5\textwidth]{16.png}

Распишем потенциал точечных зарядов:

\fc{\varphi(\vec{r})=\underset{i}{\Sigma}\frac{q_i}{|\vec{r}-\vec{r_i'}|}=:\Sigma \frac{q}{|\vec{r}-\vec{r'}|}}

\newpage

Используем разложение:

\fc{\frac{1}{|\vec{r}-\vec{r'}|}=\frac{1}{r}+(-\vec{r'})\gradd\frac{1}{r}+...=\frac{1}{r}+(-r')(-\frac{1}{r^2}\cdot\frac{\vec{r}}{r})+...=\frac{1}{r}+\frac{\vec{r}\vec{r'}}{r^3}}

получаем

\fc{\varphi=\Sigma q\frac{1}{r}+\Sigma q\vec{r'}\frac{\vec{r}}{r^3}+...=\frac{Q}{r}+\frac{\vec{d}\vec{r}}{r^3}+...}

где

\kr{Дипольный момент-} $\vec{d}:=\underset{i}{\Sigma}q_i\vec{r_i'}$

Полный заряд системы-$Q=\underset{i}{\Sigma}q_i$

Дипольный член в сферических координатах( $\vec{e_z}\uparrow\uparrow\vec{d}$ ):

\fc{\varphi(r,\theta)=\frac{d}{r^2}\cos \theta}

\subsection*{Потенциал и поле диполя}

Из прошлого пункта:

\fc{\varphi=\frac{Q}{r}+\frac{\vec{d}\vec{r}}{r^3}}

Найдем поле диполя:

\fc{\vec{E}=-\grad \varphi=-\grad \left((\vec{d}\vec{r})\frac{1}{r^3})\right)=-\grad \left(\overset{\downarrow}{(\vec{d}\vec{r})}\frac{1}{r^3}\right)-\grad \left((\vec{d}\vec{r})\overset{\downarrow}{\frac{1}{r^3}}\right)=}

\fc{=-\frac{1}{r^3}\grad (\vec{d}\vec{r})-(\vec{d}\vec{r})\grad\frac{1}{r^3}=-\frac{\vec{d}}{r^3}+3\frac{(\vec{d}\vec{r})}{r^4}\grad \vec{r}=-\frac{\vec{d}}{r^3}+3\frac{(\vec{d}\vec{r})\vec{r}}{r^5}}

Итог, \kr{поле диполя:}

\fc{\vec{E}=-\frac{\vec{d}}{r^3}+3\frac{(\vec{d}\vec{r})\vec{r}}{r^5}}
