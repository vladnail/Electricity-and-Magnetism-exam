


\newpage
\section*{21. Закон сохранения заряда. Уравнение непрерывности. Закон Джоуля-Ленца.}
 
\subsection*{Закон сохранения заряда}

Заряд:

1. Заряд является величиной инвариантной относительно переходов в различные СО;

2. Величина заряда не зависит от скорости частицы;

3. Ни в каких физических процессах $\Sigma$ количества заряда не меняется. 

\subsection*{Уравнение непрерывности}

Закон сохранения заряда:

\[\begin{array}{l|l}
    \frac{dq}{dt}=-I & \Rightarrow \underset{V}{\iiint} \frac{d\rho}{dq}dV +\underset{\delta V}{\oiint} \vec{j}d\vec{S}=0  \\
    I=\underset{\delta V}{\oiint} \vec{j}d\vec{S} & \underset{V}{\iiint}\underset{\forall V }{\left( \frac{d\rho}{dq}+\mm{div} \vec{j}  \right)}dV=0 \Rightarrow \boxed{\frac{d\rho}{dq}+\mm{div} \vec{j}=0 } \\
    q=\underset{V}{\iiint}\rho dV 
\end{array}\]

Где $\rho-$плотность заряда, $\vec{j}-$плотность потока заряда.

Если процесс стационарный, при котором токи и зардя не меняются со временем, то:

\[\mathrm{div}\vec{j} =0\]

\subsection*{Закон Джоуля-Ленца}

\[\frac{d\check{Q}}{dV} =<\vec{F}\vec{v}>n=e\vec{E}<\vec{v}>n=\vec{j}\vec{E}(\vec{j}=ne<\vec{v}>)\]

