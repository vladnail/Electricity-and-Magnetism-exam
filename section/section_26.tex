\section*{26. Ток в вакууме. Закон «трёх вторых».}
 
\textit{Вакуумный диод:}

\imc[0.8\textwidth]{52.png}

\textit{Термоэлекронная эмисия }-из нагретого электрода вылетают электроны.

\textit{Режим насыщения}-при больших напряжениях электрон эмисировал и тут же его прятнуло к положительному электроду.

\imc[0.6\textwidth]{53.png}

\[\Delta\varphi=-4\pi\rho=-4\pi n e \text{, где  } n \text{ плотность электронов, зависит от x}\]

\[I=jS.\text{Из уравнения непрерывности:} \left( \cancelto{\scriptstyle{0\text{ (стационарный режим)}}}{ \frac{\partial\rho}{\partial t}}+\mathrm{div}\vec{j}   \right)=0\Rightarrow\]

\newpage

\[\Rightarrow \mathrm{div}\vec{j}=0\Rightarrow \frac{\partial j_x}{\partial x}=0 \text{ ,т.е } j=\mathrm{const}=\frac{I}{S}   \]

\[\begin{cases}
    \varphi''=-4\pi ne & \\
    \mathrm{const}=\frac{I}{S}=nev & \Rightarrow 
    \begin{cases}
        v=\sqrt{\frac{2e}{m}}\varphi^{\frac{1}{2}} & \\
        ne=-\frac{I}{S}\frac{1}{v} &   \\
    \end{cases} \\
    \frac{mv^2}{2}=e\varphi &
\end{cases}
\]
 
\[ \text{Гран.условия:} 1)\varphi'|_{x=0}=0;2)\varphi|_{x=0}=0;3)\varphi|_{x=d}=U. \]

\[
\begin{aligned}
    \begin{array}{l|l}
        v=\sqrt{\frac{2e}{m}}\varphi^{\frac{1}{2}} & \\
        ne=-\frac{I}{S}\frac{1}{v} & 
    \end{array}
    \begin{array}{l|l}
        \Rightarrow ne=\sqrt{\frac{m}{2e}} \frac{-I}{S}  & \\
        \varphi''=-4\pi en &
    \end{array} 
    \begin{array}{ll}
        \Rightarrow \varphi '\varphi ''=4\pi \sqrt{\frac{m}{2e}}\varphi^{\frac{1}{2}} \frac{I}{S}\varphi^{-\frac{1}{2} }\varphi' &\\
        \frac{1}{2}(\varphi'^2)'=4\pi \sqrt{\frac{m}{2e}} (\varphi^{\frac{1}{2} })'\cdot 2 &\\
    \end{array}  \\
    \varphi'^2 =  \underbrace{16\pi\sqrt{\frac{m}{2e}}\frac{I}{S}\varphi^{\frac{1}{2}}}_{a}+\cancelto{\scriptstyle{0(\text{Гран.услов.})}}{C}  \\
    \varphi'=\sqrt{a}\varphi^{\frac{1}{4}}\Rightarrow  \frac{\partial\varphi}{\partial x}=\sqrt{a}\varphi^{\frac{1}{4} } \\
    \frac{4}{3}\varphi^{\frac{1}{4}}=  \sqrt{a}x +\cancelto{\scriptstyle{0(\text{Гран.услов.})}}{C}
\end{aligned}
\]

\[
\frac{4}{3}U^{\frac{3}{4}}=\sqrt[1]{a}d\Rightarrow a=\frac{16}{9} \frac{1}{d^2}U^{\frac{3}{2}}=16\pi\sqrt{\frac{m}{2e}}\frac{I}{S} \Rightarrow \boxed{I=\frac{\sqrt{2}}{9\pi}\sqrt{\frac{e}{m}}\frac{S}{d^{2}}U^{\frac{3}{2} }  }  
\]

\imc[0.75\textwidth]{54.png}