\section*{34. Векторы В и Н. Магнитная проницаемость. Полная система уравнений для
магнитного поля в среде.}

\subsection*{Векторы В и Н}

\[
\begin{aligned}
    \begin{cases}
        \mathrm{div}\vec{H}=0  \\ 
        \mathrm{rot}\vec{H}=\frac{4\pi}{c}(\vec{j}+\vec{j}_{\text{м}})
    \end{cases}
    \overset{\text{уср}}{\rightarrow}
    \begin{cases}
        \quad \mathrm{div}<\vec{H}>=0 \\
        \mathrm{rot}<\vec{H}= \frac{4\pi}{c}(\vec{j}+<\vec{j}_{\text{м}}>)
    \end{cases}
    \rightarrow
\end{aligned}
\]

\[
\begin{aligned}
    \overset{<\vec{H}>=:\vec{B}}{\rightarrow}
    \begin{cases}
        \mathrm{div}\vec{B}=0 \\
        \mathrm{rot}\vec{B}= \frac{4\pi}{c}(\vec{j}+<\vec{j}_{\text{м}}>)
    \end{cases}
    \overset{<\vec{j}_{\text{м}}>=c\cdot \mathrm{rot}\vec{M}}{\rightarrow}
    \begin{cases}
        \mathrm{div}\vec{B}=0 \\
        \mathrm{rot}(\vec{B}-4\pi\vec{M})= \frac{4\pi}{c}\vec{j}
    \end{cases}
    \rightarrow 
\end{aligned}
\]

\[
\begin{aligned}
    \overset{\vec{B}-4\pi\vec{M}=:\vec{H}}{\rightarrow}
    \boxed{\begin{cases}
        \mathrm{div}\vec{B}=0 \\
        \mathrm{rot}\vec{H}= \frac{4\pi}{c}\vec{j}
    \end{cases}
    }
\end{aligned}
\]

\[
\vec{B}=(1-4\pi\chi)\vec{H}
\]

\[
\text{(Рекумендуется просмотр прошлого вопроса для полноты ответа) }
\]

\subsection*{Магнитная проницаемость}

\[
\mu=1+4\pi\chi
\]

\subsection*{Полная система уравнений для
магнитного поля в среде}

\[
    \begin{cases}
        \mathrm{div}\vec{B}=0 \\
        \mathrm{rot}\vec{H}= \frac{4\pi}{c}\vec{j}
    \end{cases}
\]
 
