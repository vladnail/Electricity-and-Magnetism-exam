\section{Уравнение Лапласа. Разделение переменных в уравнении Лапласа в
декартовой системе координат.}

\subsection*{Уравнение Лапласа}

В декартовой системе координат

\fc{\displaystyle {\frac {\partial ^{2}\varphi}{\partial x^{2}}}+{\frac {\partial ^{2}\varphi}{\partial y^{2}}}+{\frac {\partial ^{2}\varphi}{\partial z^{2}}}=0}

В сферической системе ($r,\theta,\alpha$) координат

\fc{{\displaystyle {1 \over r^{2}}{\partial  \over \partial r}\left(r^{2}{\partial \varphi \over \partial r}\right)+{1 \over r^{2}\sin \theta }{\partial  \over \partial \theta }\left(\sin \theta {\partial \varphi \over \partial \theta }\right)+{1 \over r^{2}\sin ^{2}\theta }{\partial ^{2}\varphi \over \partial \alpha ^{2}}=0}}

В цилиндрической ($r,\alpha,z$) координат

\fc{{\displaystyle {1 \over r}{\partial  \over \partial r}\left(r{\partial \varphi \over \partial r}\right)+{\partial ^{2}\varphi \over \partial z^{2}}+{1 \over r^{2}}{\partial ^{2}\varphi \over \partial \alpha ^{2}}=0}}

\subsection*{Разделение переменных в уравнении Лапласа в
декартовой системе координат}

Предположим, что в декартовых координатах переменные разделяются -это означает, что: 
\fc{\varphi(x,y,z)=X(x)\cdot Y(y)\cdot Z(z)}

\fc{\Delta \varphi=0 \Rightarrow X''YZ+XY''Z+XYZ''=0}

\newpage


\fc{\frac{X''}{X}+\frac{Y''}{Y}+\frac{Z''}{Z}=0 \Rightarrow Const_1+C_2+C_3=0}

\fc{\frac{X''}{X}=C \Rightarrow X''=CX}

\fc{(1)X(x)=
\left\{
\begin{aligned}
\text{при C} >0 ,Ae^{\sqrt{c}x} \\
\text{при C} <0 ,Ae^{\pm i\sqrt{c}x} \\
\text{при C} =0 ,Ax+B
\end{aligned}
\right.}

При $\rho \neq0$. Допустим, что 

\fc{\rho(x,y,z)=\rho \cdot X(x)\cdot Y(y)\cdot Z(z),\text{где X,Y,Z функции вида (1)}}

Тогда 

\fc{\varphi=A\cdot X(x)\cdot Y(y)\cdot Z(z)}

\fc{A(X''YZ+XY''Z+XYZ'')=-4\pi \rho_0 XYZ\Rightarrow \frac{X''}{X}+\frac{Y''}{Y}+\frac{Z''}{Z}=-\frac{4\pi \rho_0}{A}}

\fc{C_1+C_2+C_3=-\frac{4\pi \rho_0}{A}}

Итог

\fc{\rho = p _ { 1 } + p _ { 2 } , \Delta \varphi = - 4 \pi \rho , \varphi = \varphi _ { 1 } + \varphi _ { 2 } ;}

\fc{
\left\{
\begin{aligned}
\Delta \varphi_1=-4\pi \rho_1 \\
\Delta \varphi_2=-4\pi \rho_2  \
\end{aligned}
\right.}
