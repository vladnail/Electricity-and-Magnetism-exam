
\section*{2. Дивергенция электрического поля. Распределённый заряд. Основное
уравнение электростатики, его общее решение в безграничном пространстве}

\subsection*{Дивергенция электрического поля}

Вспомним теорему Гаусса для потока $\vec{E}$ через замкнутую площадь S

\fc{\underset{\delta V}{\oiint}\vec{E}d\vec{S}=4\pi Q= \underset{V}{\iiint}
4\pi \rho dV }
а по теореме Остроградского-Гаусса 

\fc{\underset{\delta V}{\oiint}\vec{E}d\vec{S}=\underset{V}{\iiint}div
\vec{E}d\vec{V}}

следует что для $ \forall V$ :

\fc{\underset{V}{\iiint}div \vec{E}d\vec{V}=4\pi Q= \underset{V}{\iiint} 4\pi
\rho dV\Rightarrow div\vec{E}=4\pi \rho}  

\subsection*{Распределённый заряд}

Объемная плотность заряда: 

\fc{dq\overset{df}{=}\rho dV}

Поверхностная плотность:

\fc{dq\overset{df}{=}\sigma dS}

Линейная плотность:

\fc{dq\overset{df}{=}\kappa dl}

\subsection*{Основное
уравнение электростатики, его общее решение в безграничном пространстве}

В конечной области пространства с плотностью заряда $\rho(\vec{r})$, по
принципу суперпозиции скалярный потенциал этих зарядов равен:

\fc{\varphi (\vec{r})=\int \frac{\rho(\vec{r'})dV'}{|\vec{r}-\vec{r'}|}}

\imc[0.5\textwidth]{4.png} 

Представление потенциала в виде интеграла по объему, занятому зарядами, часто
называют частным решением уравнения Пуассона.

Для задачи с точечными зарядами интегральная форма не подойдёт, перейдём к
сумме. Введём функцию Дирака $\delta$, она задается следующими условиями:

1) при всех $\vec{r} \neq 0$ $\delta(\vec{r})=0$ ;

2) в точке $\vec{r}\neq 0$ имеем $\delta(\vec{r})=\infty$ ;

3) интеграл по всему пространству $\int\delta(\vec{r})dV=1$

4) $\int f(\vec{r})\delta(\vec{r}-\vec{r_0})dV=f(\vec{r_0})$

где $f(\vec{r})$ - произвольная непрерывная функция, $\vec{r_0}$
радиус-вектор некоторой фиксированной точки.

Объёмную плотность заряда расположенного в точке $\vec{r}=\vec{r_0}$ можно
перезаписать:
\fc{\rho(\vec{r})=q\delta(\vec{r}-\vec{r_0})}

подставляем в предыдущую формулу

\fc{\varphi (\vec{r})=\int
\frac{q\delta(\vec{r}-\vec{r_0})dV'}{|\vec{r}-\vec{r'}|}=\frac{q}{|\vec{r}-\vec
{r'}|}}


