\section{Электрический ток, дрейфовая скорость, подвижность. Объемная и
поверхностная плотность тока. Электропроводность. Закон Ома.}

\subsection*{Электрический ток, дрейфовая скорость, подвижность}

Скорость Ферми в Me(metal):

\fc{\upsilon_F\thicksim10^6\text{ м/с}}

\fc{\frac{4}{3}\pi \upsilon_F^3 \sim  n}

Пространство скоростей квазиэлектронов:

\imc[0.5\textwidth]{40.png} 

При $\vec{E}=0\text{, то }<\vec{\upsilon}>=0$

При конченом малом $\vec{E}\text{, то} <\vec{\upsilon}>=<\vec{\upsilon_d}>$

\fc{\vec{\upsilon}=\vec{\upsilon_0}+\frac{\vec{E}e}{m}t}

\fc{<\vec{\upsilon}>=\cancelto{0}{<\vec{\upsilon_0}}>+\frac{\vec{E}e}{m}<t>}

Где $<t>=\tau-$ время релаксации импульса.

Релаксация импульса:
\fc{<\vec{\upsilon}>=\frac{\vec{E}e}{m}\tau=\vec{\upsilon_d}\Rightarrow\frac{e\tau }{m}\vec{E}=\vec{\upsilon_d}}

Или

\fc{\vec{\upsilon_d}=\mu\vec{E}}

где $\mu=\frac{e\tau }{m}-$\kr{подвижность носителей заряда.}

\newpage

\subsection*{Объемная и поверхностная плотность тока}

Ток-это заряд в единицу времени.

\kr{Ток в проводе:}
 
\imc[0.5\textwidth]{41.png} 

\fc{I = \frac{dQ}{dt} = \frac{en\upsilon \, dt \cdot S_0}{dt}  = ne \, \vec{\upsilon} \cdot \vec{S}}

\kr{Электрический ток через} $\vec{S}:$

\imc[0.5\textwidth]{42.png} 

\fc{\begin{array}{l|l}
I\overset{df}{=}\underset{S}{\iint} \vec{j}d\vec{S} & \Rightarrow\vec{j}=ne\vec{\upsilon_d} \\ 
I=\underset{S}{\iint} ne\vec{\upsilon_d}d\vec{S}	 & \vec{j}=\rho \vec{\upsilon_d} \
\end{array}
}

\kr{Ток по поверхности:}

\imc[0.5\textwidth]{43.png} 

\fc{dI=idl \text{, где } i-\text{поверхностная плотность тока.}}

\newpage

\subsection*{Электропроводность}

\imc[0.7\textwidth]{44.png}

\fc{\vec{j}=\sigma\vec{E}\text{, где }\sigma-\kr{электропровдность(проводимость).}}

\fc{
\begin{array}{l|l}
	\vec{j}=ne\vec{\upsilon_d} &\Rightarrow\boxed{\sigma=ne\mu=\frac{ne^2\tau}{m}}\\
	\vec{\upsilon_d}=\frac{e\tau}{m}\vec{E}=\mu\vec{E} &

\end{array}
}

\fc{\sigma=ne\mu=\frac{ne^2\tau}{m}-\text{формула}\kr{ Пауля Друде.}}

\subsection*{Закон Ома}

Закон Ома в дифференциальной форме: $\vec{j}=\sigma\vec{E}$

\imc[0.7\textwidth]{45.png}



\fc{jV=\sigma EV}

\fc{jSl=\sigma ElS\Rightarrow I\cdot l=\sigma US\Rightarrow U=\frac{1}{\sigma}\frac{l}{S}I}
Со школы: $U=\rho\frac{l}{S}I$,

\fc{\rho=\frac{1}{\sigma}-\text{удельное сопротивление.}}







