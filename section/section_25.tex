\section*{25. Электродвижущая сила. Электрические цепи. Законы Кирхгофа.}
 
\subsection*{Электродвижущая сила}

\imc[0.4\textwidth]{46.png}

$\vec{F}-$стороняя сила, действующая на единичный заряд.

Сторонее "электрическое" поле $\vec{E_c}= \frac{\vec{F_c}}{q}.$

\[\vec{F}=\vec{F_e}+\vec{F_c}=q(\vec{E}+\vec{E_c})\]

\[\underset{L}{\oint}\vec{F}d\vec{l}=q\left( \underset{L}{\oint} \underset{\rightarrow 0}{\vec{E}d\vec{l}} + \underset{L}{\oint} \vec{E_c}d\vec{l} \right)=q\underset{L}{\oint} \vec{E_c}d\vec{l}\]

\[-jV=\sigma V(E-E_c)\Rightarrow-jSl=\sigma S(El-E_cl)\]

\[
\begin{array}{l|l}
    -jSl=\sigma S(El-E_cl) & \Rightarrow -Il=\sigma S(U-\varepsilon)\\
    \varepsilon\overset{df}{=}\underset{L}{\oint} \vec{E_c}d\vec{l} & U=-\frac{q}{\sigma}\frac{l}{S}I+\varepsilon     
\end{array}
\]

Напоминание:$\frac{q}{\sigma}=\rho, \rho\frac{l}{S}=r-$внутрение сопротивление источника.

\begin{center}
    Итог:\boxed{U=\varepsilon-Ir}
\end{center}

\newpage

\subsection*{Электрические цепи и законы Кирхгофа}

\textit{Первый закон Кирхгофа (закон токов):}
\[
\sum_{i} I_i = 0,
\]
где:  
\begin{itemize}
    \item \( I_i \) — сила тока, текущего в узел (положительное значение для входящих токов и отрицательное для выходящих).
\end{itemize}

\textit{Физический смысл:}  
Сумма токов, входящих в узел электрической цепи, равна сумме токов, выходящих из него. Это выражение закона сохранения заряда.

\textit{Второй закон Кирхгофа (закон напряжений):}
\[
\sum_{k} \mathcal{E}_k - \sum_{j} I_j R_j = 0,
\]
где:  
\begin{itemize}
    \item \( \mathcal{E}_k \) — \textit{электродвижущая сила (ЭДС)} источников в контуре, измеряется в вольтах (\( \text{В} \));
    \item \( I_j \) — \textit{сила тока} в ветви, измеряется в амперах (\( \text{А} \));
    \item \( R_j \) — \textit{сопротивление} элементов в контуре, измеряется в омах (\( \Omega \)).
\end{itemize}

Физический смысл:  
Алгебраическая сумма ЭДС и падений напряжений на всех элементах любого замкнутого контура электрической цепи равна нулю. Это следствие закона сохранения энергии.



\textit{Пример применения:}

\imc[0.8\textwidth]{51.png}

\newpage

1. Для узла:  
Если три тока подходят к узлу, а два выходят, то:
\[
I_1 + I_2 - I_3 - I_4 = 0.
\]

2. \textit{Для замкнутого контура:}  
Для цепи с ЭДС \( \mathcal{E}_1, \mathcal{E}_2  \), сопротивлениями \( R_1, R_2, R_3 \), и токами \( I_1, I_2, I_3 \):
\[
\mathcal{E}_1+\mathcal{E}_2 - I_1 R_1 - I_2 R_2-I_3 R_3 = 0.
\]


