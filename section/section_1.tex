
	\section{Закон Кулона. Напряжённость электрического поля. Принцип
суперпозиции.Поток электрического поля. Теорема Гаусса.}

\subsection*{Закон Кулона}

Это — экспериментально установленный закон силового взаимодействия двух
точечных заряженных тел, неподвижных относительно рассматриваемой системы
отсчета, согласно которому:

\[\boxed{\vec{F_k}=\frac{q_1q_2}{r_{12}^2}\frac{\vec{r_{12}}}{r_{12}}}\]

\imc[0.5\textwidth]{2.png} 

Введем понятие напряженности:

\[\boxed{\vec{E}_1(\vec{r}_2) = \frac{q_1}{r_{12}^2} \frac{\vec{r}_{12}}{r_{12}}}\]

тогда силу Кулона можно перезаписать в виде:

\[\vec{F}_{12} = q_2 \vec{E}_1(\vec{r}_2)\]

\subsection*{Напряжённость электрического поля}

В общем виде напряженность имеет вид:
\[\boxed{\vec{E}(\vec{r}) = \frac{q}{|\vec{r} - \vec{r}_0|^2} \frac{\vec{r} -
\vec{r}_0}{|\vec{r} - \vec{r}_0|}}\]

\imc[0.5\textwidth]{1.png} 

\subsection*{Принцип суперпозиции}

Электрическое поле от системы зарядов равно сумме электрических полей от её
составляющих:

\fc{\boxed{\vec{E}(\vec{r}) = \sum_i \vec{E}_i(\vec{r})} = \sum_i \frac{q_i}{|\vec{r} -
\vec{r}_i|^2}\frac{ \vec{r} - \vec{r}_i}{|\vec{r} - \vec{r}_i|}}

\subsection*{Поток электрического поля}

Если у нас имеется некоторая конечная поверхность S, то поток
через эту поверхность вычисляется как поверхностный интеграл

\fc{\boxed{\text{Ф}= E_ndS}}

\subsection*{Теорема Гаусса}

\kr{Теорема Гаусса:}Поток вектора $\vec{E}$ через любую замкнутую
поверхность определяется суммарным зарядом Q, находящимся внутри этой
поверхности, и равняется 4$\pi$Q:

\fc{\boxed{\oint_SE_nds=4\pi Q}}



