\section{Диэлектрики. Связанный заряд. Вектор поляризации. Электрическое поле
в диэлектрике. Вектор индукции. Диэлектрическая проницаемость.}

\subsection*{Диэлектрики}

\kr{Неполярный диэлеткрик}(к примеру $H_2,O_2$):

\imc[0.3\textwidth]{29.png}

\fc{\vec{d_i}=0}

Под действием поля $\vec{E}$ происходит смещение электронного облака
 $\text{ и}<\vec{d_i}\neq0>$

\kr{Ионный диэлектрик:}

\imc[0.4\textwidth]{30.png}

\fc{\vec{d_i}=0}
\fc{\text{Eсли }\vec{E}=0\Rightarrow <\vec{d_i}>=0}
\fc{\text{Eсли }\vec{E}\neq0\Rightarrow U=-\vec{d}\vec{E}\text{, }<\vec{d}>\neq0}

\newpage

\kr{Полярный диэлектрик:}

\imc[0.25\textwidth]{31.png}

\fc{\vec{d_i}\neq0}
\fc{\text{Eсли }\vec{E}=0\Rightarrow <\vec{d}>=0\text{, }U=-\vec{d}\vec{E}}

\subsection*{Связанный заряд и Вектор поляризации}

\imc[0.4\textwidth]{32.png}

\fc{\iiint<\rho_c>dV=0, \rho_c\text{-связанные заряды }}

\fc{<\rho_c>=\frac{1}{\Delta V}\underset{V}{\iiint}\rho_c(\vec{r}+\vec{\xi})d\vec{\xi}\text{, }\Delta V\sim 10^{-6}\text{см}^3}

Определение\kr{ вектора поляризации:}

\fc{\boxed{<\rho_c>=:-\mm{div}\vec{P}}}

Вне тела $\vec{P}=0$

\newpage

\imc[0.3\textwidth]{33.png}

По формуле Остроградского-Гаусса (1):

\fc{\iiint<\rho_c>dV=-\iiint\mm{div}\vec{D}dV\overset{(1)}{=}-\oiint\vec{P}d\vec{S}\Rightarrow}
\fc{\Rightarrow \boxed{P_n=-\sigma_c}}

Связь вектора поляризации и дипольного момента:

\fc{\vec{d}=\iiint <\rho_c>\vec{r}dV=-\iiint\vec{r}(\grad\vec{P})dV=-\iiint\grad\underset{\rightarrow0}{(\vec{P}\vec{r})}dV+\iiint(\vec{r}\grad)\vec{P}dV=}
\fc{\grad(\vec{P}\vec{r})=\vec{r}(\grad\vec{P})+(\vec{r}\grad)\vec{P}\text{ , }(\vec{r}\grad)\vec{P}=\bigg(x\frac{\partial}{\partial x}+y\frac{\partial}{\partial y}+z\frac{\partial}{\partial z}\bigg)(P_1,P_2,P_3)=\vec{P}}
\fc{=\iiint\vec{P}dV}
\fc{\Rightarrow\boxed{\vec{P}=n<\vec{d}>}}

Вектор поляризации $\vec{P}$ равен дипольному моменту единицы объема поляризованного диэлектрика.

\fc{\boxed{\vec{P}=\chi\vec{E}}, \chi \text{-поляризуемость.}}

\subsection*{Электрическое поле в диэлектрике и Вектор индукции и Диэлектрическая проницаемость}

\fc{
\begin{cases}
\mm{div}<\vec{E}>=4\pi(\rho+<\rho_c>) \qquad\qquad  \mm{div}(\vec{E}+4\pi\vec{P})=4\pi\rho\Rightarrow \mm{div}\vec{D}=4\pi\rho\\
\mm{rot}\vec{E}=0 \qquad\qquad\qquad\qquad\qquad\quad\quad \vec{D}:=\vec{E}+4\pi\vec{P}\text{(нет физического смысла)}
\end{cases}}

$<\vec{E}>:=\vec{E}$-напряженность электрического поля,

$\vec{D}$-\kr{вектор индукции электрического тока}.

По теореме Гаусса:

\fc{\oiint\vec{D}d\vec{S}=4\pi Q }

\fc{\underset{\text{Г}}{\oint}\vec{E}d\vec{l}=0\Rightarrow\vec{E}=-\grad\varphi\text{, }\varphi=-\int\vec{E}d\vec{l}}
\fc{\vec{D}=\vec{E}+4\pi\chi\vec{E}=(1+4\pi\chi)\vec{E}=\varepsilon\vec{E}}

$\varepsilon$\kr{-диэлектрическая проницаемость}($\varepsilon\geq1$)

\fc{\boxed{\vec{P}=\frac{\vec{D}-\vec{E}}{4\pi}}}

\fc{\rho_c=-\grad\bigg(\frac{\vec{D}-\vec{E}}{4\pi}\bigg)=-\grad\bigg(\frac{\varepsilon-1}{4\pi}\vec{E}\bigg)=\frac{1-\varepsilon}{4\pi}\grad\vec{E}-\vec{E}\frac{\grad\varepsilon}{4\pi}}

\kr{Итого имеем:}

\fc{
\begin{cases}
\mm{div}\vec{D}=4\pi\rho \qquad\qquad \oiint\vec{D}d\vec{S}=4\pi Q \\
\mm{rot}\vec{E}=0 \qquad\qquad\quad \oint\vec{E}d\vec{S}=0\\
\vec{D}=\varepsilon\vec{E}
\end{cases}}