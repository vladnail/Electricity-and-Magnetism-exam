\section{Граничные условия для нормальной и тангенциальной компонент
электрического поля. Поверхностная плотность зарядов. Поле вблизи
поверхности металлов. Граничные условия для электрического поля,
выраженные через его скалярный потенциал.}

\subsection*{Граничные условия для нормальной и тангенциальной компонент
электрического поля}

\kr{Тангенциальная компонента}

\fc{\mm{rot}\vec{E}=0 \Rightarrow \oint \vec{E}d\vec{l}\leftarrow \text{интегральная форма}}

По теореме Стокса

\fc{0=\underset{(\forall )S}{\iint} \mm{rot}\vec{E}dS=\underset{(\forall )S}{\oint }\vec{E}d\vec{l}}

\imc[0.5\textwidth]{10.png} 


\fc{\oint \vec{E}d\vec{l}=E_x|_1\cdot l-E_x|_2\cdot l\Rightarrow E_x|_1=E_x|_2}

или же 

\fc{\boxed{E_{1\tau}|=E_{2\tau}|}}

\kr{Нормальная компонента}

\fc{\mm{div}\vec{E}=4\pi \rho \Rightarrow \underset{(\forall)S}{\oiint}\vec{E}d\vec{S}=4\pi Q \Rightarrow}

\fc{\Rightarrow \underset{(\forall)V}{\iiint} \mm{div}\vec{E}dV=4\pi \underset{(\forall)V}{\iiint} \rho dV \Rightarrow \underset{(\forall) S}{\oiint}\vec{E}d\vec{S}=4\pi Q }

\imc[0.5\textwidth]{11.png}

\fc{E_{1n}|\cdot S-E_{2n}|\cdot S=4\pi Q=4\pi \rho S}

или же 

\fc{\boxed{E_{1n}| -E_{2n}| =4\pi \rho }}

\subsection*{Поверхностная плотность зарядов(???)}

\fc{dq\overset{df}{=}\sigma dS}

\subsection*{Поле вблизи поверхности металлов}

Надо доказать что поле вблизи металлов равно

\fc{\vec{E}=4\pi \sigma \vec{n}}

Рассмотрим тангенсальную и нормальную компоненту поля $\vec{E}$ на границе металла
 
\imc[0.4\textwidth]{12.png}

Если в проводнике имеется электрическое поле, то по нему течёт ток. Следовательно, для электростатических явлений электрическое поле внутри проводника $E_{1n}=0$ отсюда 

\fc{E_{2n}=4\pi \sigma}

Снаружи металла поле $E_{2\tau }=0$ и из граничных условий  

\fc{E_{1\tau}=0}

Итоговое поле равно 

\fc{\boxed{\vec{E}_{2n}=4\pi \sigma \vec{n}}}

Что и требовалось доказать.

\subsection*{Граничные условия для электрического поля,
выраженные через его скалярный потенциал}

Можно рассмотреть две точки А и B с одной стороны поверхности и C,D с другой стороны. Найдем напряжение между парами этих  точек:

%Использовал пакет TikZ для пробы
\begin{center}
\begin{tikzpicture}
    % Ось X
    \draw[->] (-3,0) -- (3,0) node[right] {$x$};

    % Верхний вектор от A к B
    \node[above] at (-2,1) {$A$};
    \node[above] at (2,1) {$B$};
    \draw[->, thick] (-2,1) -- (2,1);

    % Нижний вектор от C к D
    \node[below] at (-2,-1) {$C$};
    \node[below] at (2,-1) {$D$};
    \draw[->, thick] (-2,-1) -- (2,-1);
\end{tikzpicture}
\end{center}

Из граничных условий, что $E_{\tau}-$непрерывно следует, что:

\fc{E_{\text{AB}|} = E_{\text{CD}}|} 

потенциал можно выразить через напряженность так:

\fc{{E}=-\mm{grad}\varphi}

отсюда получаем, что $\varphi_{\text{AB}}|=\varphi_{\text{CD}}|\Rightarrow \varphi|-\text{непрерывно}$