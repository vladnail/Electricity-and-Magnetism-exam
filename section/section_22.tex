\newpage

\section{Уравнения постоянного тока. Граничные условия.}
 
\subsection*{Уравнения постоянного тока}



\[I=\frac{U}{R} \]
Ниже ипользованны формулы, которые должны были быть ведены в этом пункте для дальнейшего вывода форумул.
\subsection*{Граничные условия}
\[
\begin{array}{l|l}
    \mathrm{div}\vec{j}=0 \quad (1)& (1)\Rightarrow\underset{V}{\iiint}\mathrm{div}\vec{j}dV=\underset{\delta V}{\oiint}\vec{j}d\vec{S}=0 \quad (1') \\
    
    \begin{array}{l|l}
        \vec{j}=\sigma\vec{E} &\Rightarrow \mathrm{rot} \frac{\vec{j}}{\sigma}=0 \quad (2)  \\
        \mathrm{rot}\vec{E}=0 
    \end{array}    & (2)\Rightarrow \underset{S}{ \iint}\mathrm{rot} \frac{\vec{j}}{\sigma}dS=\underset{\delta S}{\oint} \frac{\vec{j}}{\sigma} d\vec{l}=0\quad (2')  
    
\end{array}
\]

\textit{Нормальная состовляющая}

\imc[0.4\textwidth]{47.png}

\[(1')=j_{1n}|S-j_{2n}|S=0\Rightarrow j_{1n}=j_{2n} \]

\[\boxed{j_n-\text{непр.}}\]

\textit{Тангенсальная состовляющая}

\imc[0.45\textwidth]{48.png}

\[(2')=\Rightarrow \frac{j_{1\tau}}{\sigma_1}\bigg|l-\frac{j_{2\tau}}{\sigma_2}\bigg|l=0\Rightarrow \frac{j_{1\tau}|}{\sigma_1}=\frac{j_{2\tau}|}{\sigma_2}\]

\[\boxed{\frac{j_{\tau}}{\sigma} -\text{непр.}}\]
