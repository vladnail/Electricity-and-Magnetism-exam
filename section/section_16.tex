\section*{16. Уравнения электрического поля в диэлектрике. Граничные условия.}

\subsection*{Уравнения электрического поля в диэлектрике}

\fc{
\text{Усреднение:}
\begin{cases}
\text{div}\, \vec{E} = 4\pi (\langle \rho_c \rangle + \rho) & \quad \langle \rho_c \rangle = - \text{div}\, \vec{P} \\
\text{rot}\, \vec{E} = 0 & \quad \vec{P} = \chi \langle \vec{E} \rangle
\end{cases}
}
\fc{
\text{Обозначения:}
\begin{array}{rl}
<\vec{E}>=:\vec{E} 
 \begin{array}{rl}
\end{array}   \\
\vec{E}+4\pi\vec{P}=:\vec{D}
\end{array}
}
\fc{
\begin{cases}
\mm{div}\vec{D}=4\pi\rho \\
\mm{rot}\vec{E}=0 \qquad \vec{E}=-\grad\varphi	
\end{cases}
}

\fc{\text{Вектор индукции: }\vec{D}=\vec{E}+4\pi\vec{P}=(1+4\pi\chi)\vec{E}=\varepsilon\vec{E}}
\newpage

\subsection*{Граничные условия}

\kr{Тангенсальная компонента:}

\imc[0.4\textwidth]{34.png}

\fc{\begin{cases}
\mm{rot}\vec{E}=0 \\
\underset{L}{\oint}\vec{E}d\vec{l}=0 \qquad \Rightarrow \boxed{E_{1\tau}|=E_{2\tau}|}
\end{cases}
}

То есть тангенсальная компонента вектора напряжённости электрического поля на границе непрерывна, а так же:

\fc{\varphi_1|=\varphi_2|}

\kr{Нормальная компонента:}

\imc[0.55\textwidth]{35.png}

\fc{\begin{cases}
\mm{div}\vec{D}=4\pi\rho \\
\underset{S}{\oiint}\vec{D}d\vec{S}=4\pi Q \qquad \Rightarrow D_{1n}|-D_{2n}|=4\pi\sigma\text{ или } \boxed{\varepsilon_1E_{1n}-\varepsilon_2E_{2n}=4\pi\sigma}
\end{cases}
}

То есть нормальная компонента вектора
индукции электрического поля
терпит разрыв.