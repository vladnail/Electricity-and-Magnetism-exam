\documentclass[a4paper,12pt]{article}
\usepackage[utf8]{inputenc}
\usepackage[russian]{babel}
\usepackage{physics}
\usepackage{amsmath, amssymb}
\usepackage{graphicx} % Для работы с изображениями
\usepackage{geometry}
\geometry{top=2cm, bottom=2cm, left=2.5cm, right=2.5cm}

% Увеличиваем размер основного текста
\renewcommand{\normalsize}{\fontsize{14}{17}\selectfont}

% Команда для вопросов
\newcommand{\qn}[1]{
    \vspace{1em}
    \noindent\textbf{\Large #1} % Увеличенный шрифт
    \vspace{0.5em}
    \addcontentsline{toc}{section}{#1} % Добавление в оглавление
}

% Команда для подглав
\newcommand{\tc}[1]{
    \vspace{0.5em}
    \noindent\textbf{\large #1}
    \vspace{0.3em}
}

% Команда для курсива
\newcommand{\kr}[1]{\textit{#1}}

% Команда для вставки математического выражения
\newcommand{\fc}[1]{\[#1\]}

% Команда для вставки математического режима с $$ 
\newcommand{\mdd}[1]{$#1$}

% Команда для вставки математического
\newcommand{\mm}[1]{\mathrm{#1}}

%Настраиваемый вид замкнутого тройного интеграла(при умножении, в дроби)
\newcommand{\oiiint}{%
  \mathchoice%
    {\mathop{\raisebox{0.2ex}{\scalebox{1.5}[1]{\(\bigcirc\)}}}%
     \!\!\!\!\!\!\!\!\!\!\!\!\!\!\int\!\!\!\int\!\!\!\int}% Displaystyle
    {\mathop{\raisebox{0.2ex}{\scalebox{1.2}[1]{\(\bigcirc\)}}}%
     \!\!\!\!\!\!\!\!\!\!\!\!\int\!\!\!\int\!\!\!\int}% Textstyle
    {\mathop{\raisebox{0.1ex}{\scalebox{1}[1]{\(\bigcirc\)}}}%
     \!\!\!\!\!\!\!\!\!\!\int\!\!\!\int\!\!\!\int}% Scriptstyle
    {\mathop{\raisebox{0.05ex}{\scalebox{0.8}[1]{\(\bigcirc\)}}}%
     \!\!\!\!\!\!\!\!\!\int\!\!\!\int\!\!\!\int}% Scriptscriptstyle
}

%Настраиваемый вид замкнутого двойного интеграла(при умножении, в дроби)
\newcommand{\oiint}{%
  \mathchoice%
    {\mathop{\raisebox{0.1ex}{\scalebox{1.7}[0.7]{\(\bigcirc\)}}}%
     \!\!\!\!\!\!\!\!\!\!\!\int\!\!\!\int}% Displaystyle
    {\mathop{\raisebox{0.2ex}{\scalebox{1.}[0.6]{\(\bigcirc\)}}}%
     \!\!\!\!\!\!\!\int\!\!\!\int}% Textstyle
    {\mathop{\raisebox{0.1ex}{\scalebox{1}[0.8]{\(\bigcirc\)}}}%
     \!\!\!\!\!\!\!\!\!\int\!\!\!\int}% Scriptstyle
    {\mathop{\raisebox{0.05ex}{\scalebox{0.8}[0.8]{\(\bigcirc\)}}}%
     \!\!\!\!\!\!\!\!\int\!\!\!\int}% Scriptscriptstyle
}

% уменьшенный и сдвинутый градиент
\newcommand{\gradd}{\hspace{0.2cm}\scriptsize \nabla}   

  % Команда для изображения в центре
\newcommand{\imc}[2][0.7\textwidth]{%
    \begin{figure}[h!]
        \centering
        \includegraphics[width=#1]{#2}
    \end{figure}%
}

% Команда для изображения слева
\newcommand{\iml}[2][0.7\textwidth]{%
    \begin{figure}[h!]
        \raggedright
        \includegraphics[width=#1]{#2}
    \end{figure}%
}

% Команда для увеличения размера формул
\newcommand{\ds}[1]{\displaystyle #1}

%%%%%%%%%%%%%%%%%%%%%%%%%%%%%%%%%%%%%%%

\begin{document}

% Титульный лист
\begin{titlepage}
    \begin{center}
        \textbf{\large Министерство науки и высшего образования}\\
        \textbf{\largeРоссийской Федерации} \\
        \textbf{\large Федеральное государственное автономное образовательное
учреждение высшего образования} \\
        \textbf{\large «Новосибирский государственный университет» } \\
        \vspace{1em}
        \textbf{\large Факультет фундаментальных исследований} \\
        \vspace{5em}
        \textbf{\Large Дисциплина: Электромагнетизм} \\
        \vspace{2em}
        \textbf{\Large Зимняя сессия } \\
        \vspace{25em}
        \textbf{\large Приговор будет исполнен: 11.01.2024}\\
        \vspace{1em}
        \textbf{\kr {Я не несу ответственности за возможные ошибки или некорректность предоставленных ответов на билеты. Используйте их только как вспомогательный материал и обязательно сверяйтесь с официальными источниками.}  }
    \end{center}
\end{titlepage}

% Основной текст
\section*{Ответы на вопросы билета}

\qn{1. Закон Кулона. Напряжённость электрического поля. Принцип
суперпозиции.Поток электрического поля. Теорема Гаусса.}

\tc{Закон Кулона}

Это — экспериментально установленный закон силового взаимодействия двух
точечных заряженных тел, неподвижных относительно рассматриваемой системы
отсчета, согласно которому:

\[\vec{F_k}=\frac{q_1q_2}{r_{12}^2}\frac{\vec{r_{12}}}{r_{12}}\]

\imc[0.5\textwidth]{2.png} 

Введем понятие напряженности:

\[\vec{E}_1(\vec{r}_2) = \frac{q_1}{r_{12}^2} \frac{\vec{r}_{12}}{r_{12}}\]

тогда силу Кулона можно перезаписать в виде:

\[\vec{F}_{12} = q_2 \vec{E}_1(\vec{r}_2)\]

\newpage


\tc{Напряжённость электрического поля}

В общем виде напряженность имеет вид:
\[\vec{E}(\vec{r}) = \frac{q}{|\vec{r} - \vec{r}_0|^2} \frac{\vec{r} -
\vec{r}_0}{|\vec{r} - \vec{r}_0|}\]

\imc[0.5\textwidth]{1.png} 

\tc{Принцип суперпозиции}

Электрическое поле от системы зарядов равно сумме электрических полей от её
составляющих:

\fc{\vec{E}(\vec{r}) = \sum_i \vec{E}_i(\vec{r}) = \sum_i \frac{q_i}{|\vec{r} -
\vec{r}_i|^2}\frac{ \vec{r} - \vec{r}_i}{|\vec{r} - \vec{r}_i|}}

\tc{Поток электрического поля}

Если у нас имеется некоторая конечная поверхность S, то поток
через эту поверхность вычисляется как поверхностный интеграл

\fc{\text{Ф}= E_ndS}

\tc{Теорема Гаусса}

\kr{Теорема Гаусса:}Поток вектора \mdd{\vec{E}} через любую замкнутую
поверхность определяется суммарным зарядом Q, находящимся внутри этой
поверхности, и равняется 4\mdd{\pi}Q:

\fc{\oint_SE_nds=4\pi Q}

\newpage


\qn{2. Дивергенция электрического поля. Распределённый заряд. Основное
уравнение электростатики, его общее решение в безграничном пространстве}

\tc{Дивергенция электрического поля}

Вспомним теорему Гаусса для потока \mdd{\vec{E}} через замкнутую площадь S

\fc{\underset{\delta V}{\oiint}\vec{E}d\vec{S}=4\pi Q= \underset{V}{\iiint}
4\pi \rho dV }
а по теореме Остроградского-Гаусса 

\fc{\underset{\delta V}{\oiint}\vec{E}d\vec{S}=\underset{V}{\iiint}div
\vec{E}d\vec{V}}

следует что для \mdd{ \forall V} :

\fc{\underset{V}{\iiint}div \vec{E}d\vec{V}=4\pi Q= \underset{V}{\iiint} 4\pi
\rho dV\Rightarrow div\vec{E}=4\pi \rho}  

\tc{Распределённый заряд}

Объемная плотность заряда: 

\fc{dq\overset{df}{=}\rho dV}

Поверхностная плотность:

\fc{dq\overset{df}{=}\sigma dS}

Линейная плотность:

\fc{dq\overset{df}{=}\kappa dl}

\tc{Основное
уравнение электростатики, его общее решение в безграничном пространстве}

В конечной области пространства с плотностью заряда \mdd{\rho(\vec{r})}, по
принципу суперпозиции скалярный потенциал этих зарядов равен:

\fc{\varphi (\vec{r})=\int \frac{\rho(\vec{r'})dV'}{|\vec{r}-\vec{r'}|}}

\imc[0.5\textwidth]{4.png} 

Представление потенциала в виде интеграла по объему, занятому зарядами, часто
называют частным решением уравнения Пуассона.

Для задачи с точечными зарядами интегральная форма не подойдёт, перейдём к
сумме. Введём функцию Дирака \mdd{\delta}, она задается следующими условиями:

1) при всех \mdd{\vec{r} \neq 0} \mdd{\delta(\vec{r})=0} ;

2) в точке \mdd{\vec{r}\neq 0} имеем \mdd{\delta(\vec{r})=\infty} ;

3) интеграл по всему пространству \mdd{\int\delta(\vec{r})dV=1}

4) \mdd{\int f(\vec{r})\delta(\vec{r}-\vec{r_0})dV=f(\vec{r_0})}

где \mdd{f(\vec{r})} - произвольная непрерывная функция, \mdd{\vec{r_0}}
радиус-вектор некоторой фиксированной точки.

Объёмную плотность заряда расположенного в точке \mdd{\vec{r}=\vec{r_0}} можно
перезаписать:
\fc{\rho(\vec{r})=q\delta(\vec{r}-\vec{r_0})}

подставляем в предыдущую формулу

\fc{\varphi (\vec{r})=\int
\frac{q\delta(\vec{r}-\vec{r_0})dV'}{|\vec{r}-\vec{r'}|}=\frac{q}{|\vec{r}-\vec
{r'}|}}

\newpage


\qn{3. Циркуляция и ротор электрического поля. Теорема Стокса. Электрический
потенциал. Работа электрического поля. Потенциал точечного заряда.}

\tc{Циркуляция и ротор электрического поля}

Циркуляция векторного поля \mdd{\vec{E}} вдоль контура L

\fc{\underset{L}{\int} \vec{E} d\vec{l}} 

а по замкнутому контуру

\fc{\underset{L}{\oint} \vec{E} d\vec{l}=0} 

или в дифференциальной форме 

\fc{\mm{rot}\vec{E}=0}

\imc[0.5\textwidth]{5.png} 

Как следствие из теоремы о циркуляции \mdd{\vec{E}} работа при перемещении
заряда из одной точки поля в другую не зависит от формы траектории движения.

\newpage


\tc{Теорема Стокса}

\fc{\underset{\delta
S}{\oint}\vec{E}d\vec{l}=\underset{S}{\iint}\mm{rot}\vec{E}d\vec{S}}

\imc[0.5\textwidth]{6.png}

\fc{(\mm{rot}\vec{E})_{\vec{n}'}=\frac{\mm{rot}\vec{E}}{\frac{1}{\mm{cos}\theta
}}}

где \mdd{\frac{1}{\mm{cos}\theta}=\frac{dS_{\vec{n}'}}{dS_{\vec{n}}}}, получим

\fc{(\mm{rot}\vec{E})_{\vec{n}'}=\mm{rot}\vec{E}\cdot\mm{cos}\theta} 


\tc{Электрический потенциал}

Рассмотрим скаляроное поле

\imc[0.5\textwidth]{7.png}

\fc{\varphi(\vec{r}\overset{df}{=}\int_{\vec{r}}^{\vec{r_0}}\vec{E}d\vec{l}}

Чтобы определение было корректным, нужно чтобы этот интеграл не зависел от
формы \mdd{L}.

Доказательство:

\fc{\mm{rot}\vec{E}=0 \Rightarrow \forall L
\underset{L}{\oint}\vec{E}d\vec{l}=0}

Запишем выражение при обходе \mdd{L-L'} - сначала идем по контуру \mdd{L}, а
потом обратно по контуру \mdd{L'} : 

\fc{\underset{L-L'}{\oint}\vec{E}d\vec{l}=0=\underset{L}{\oint}\vec{E}d\vec{l}
-\underset{L'}{\oint}\vec{E}d\vec{l}=0}

Такие поля называются потенциальными.

Доказали.

Еще свойства потенциала: 

\imc[0.35\textwidth]{8.png}
   
\fc{\varphi(\vec{r})=\int \vec{E}(\vec{r})d\vec{r} + \underset{L}{\int}
\vec{E}d\vec{l} }

\fc{\text{и}}

\fc{\varphi(\vec{r}+d\vec{r_0})=\underset{L}{\int} \vec{E}d\vec{l}}
 
отсюда получим

\fc{\varphi(\vec{r}+d\vec{r_0})-\varphi(\vec{r})=-\vec{E}(\vec{r})d\vec{r}}

\fc{\text{так же используем}}

\fc{d\varphi=d\vec{r}\gradd \varphi }

отсюда получим

\fc{\forall d\vec{r} , \vec{E}(\vec{r})d\vec{r}=-\grad \varphi d\vec{r} \Rightarrow \vec{E}=-\mm{grad} \varphi=-\grad \varphi }

\fc{\mm{rot}\vec{E}=0=-[\grad \times \grad \varphi]}  

\newpage

\tc{Работа электрического поля}

\fc{A=\underset{L}{\int}\vec{F}d\vec{l}=\underset{L}{\int}q\vec{E}d\vec{l}=q\left[\int_{\vec{r_1}}^{\vec{r_0}}\vec{E}d\vec{l}- \int_{\vec{r_2}}^{\vec{r_1}}\vec{E}d\vec{l}   \right]=q(\varphi_2-\varphi_2)=qU}

\tc{Потенциал точечного заряда}

\fc{\varphi(r) = \frac{q}{r}}

или в общем виде

\fc{\varphi(\vec{r}) = \underset{i}{\Sigma} \frac{q_i}{|\vec{r} - \vec{r}_i|}}

\qn{4. Уравнение Лапласа. Разделение переменных в уравнении Лапласа в
декартовой системе координат.}

\tc{Уравнение Лапласа}

В декартовой системе координат

\fc{\displaystyle {\frac {\partial ^{2}\varphi}{\partial x^{2}}}+{\frac {\partial ^{2}\varphi}{\partial y^{2}}}+{\frac {\partial ^{2}\varphi}{\partial z^{2}}}=0}

В сферической системе (\mdd{r,\theta,\alpha}) координат

\fc{{\displaystyle {1 \over r^{2}}{\partial  \over \partial r}\left(r^{2}{\partial \varphi \over \partial r}\right)+{1 \over r^{2}\sin \theta }{\partial  \over \partial \theta }\left(\sin \theta {\partial \varphi \over \partial \theta }\right)+{1 \over r^{2}\sin ^{2}\theta }{\partial ^{2}\varphi \over \partial \alpha ^{2}}=0}}

В цилиндрической (\mdd{r,\alpha,z}) координат

\fc{{\displaystyle {1 \over r}{\partial  \over \partial r}\left(r{\partial \varphi \over \partial r}\right)+{\partial ^{2}\varphi \over \partial z^{2}}+{1 \over r^{2}}{\partial ^{2}\varphi \over \partial \alpha ^{2}}=0}}

\tc{Разделение переменных в уравнении Лапласа в
декартовой системе координат}

Предположим, что в декартовых координатах переменные разделяются -это означает, что: 
\fc{\varphi(x,y,z)=X(x)\cdot Y(y)\cdot Z(z)}

\fc{\Delta \varphi=0 \Rightarrow X''YZ+XY''Z+XYZ''=0}

\newpage


\fc{\frac{X''}{X}+\frac{Y''}{Y}+\frac{Z''}{Z}=0 \Rightarrow Const_1+C_2+C_3=0}

\fc{\frac{X''}{X}=C \Rightarrow X''=CX}

\fc{(1)X(x)=
\left\{
\begin{aligned}
\text{при C} >0 ,Ae^{\sqrt{c}x} \\
\text{при C} <0 ,Ae^{\pm i\sqrt{c}x} \\
\text{при C} =0 ,Ax+B
\end{aligned}
\right.}

При \mdd{\rho \neq0}. Допустим, что 

\fc{\rho(x,y,z)=\rho \cdot X(x)\cdot Y(y)\cdot Z(z),\text{где X,Y,Z функции вида (1)}}

Тогда 

\fc{\varphi=A\cdot X(x)\cdot Y(y)\cdot Z(z)}

\fc{A(X''YZ+XY''Z+XYZ'')=-4\pi \rho_0 XYZ\Rightarrow \frac{X''}{X}+\frac{Y''}{Y}+\frac{Z''}{Z}=-\frac{4\pi \rho_0}{A}}

\fc{C_1+C_2+C_3=-\frac{4\pi \rho_0}{A}}

Итог

\fc{\rho = p _ { 1 } + p _ { 2 } , \Delta \varphi = - 4 \pi \rho , \varphi = \varphi _ { 1 } + \varphi _ { 2 } ;}

\fc{
\left\{
\begin{aligned}
\Delta \varphi_1=-4\pi \rho_1 \\
\Delta \varphi_2=-4\pi \rho_2  \
\end{aligned}
\right.}



\newpage


\qn{5. Уравнение Лапласа. Разделение переменных в уравнении Лапласа в
сферической системе координат.}

\tc{Уравнение Лапласа(повтор)}

\tc{Разделение переменных в уравнении Лапласа в
сферической системе координат}

Пусть \mdd{\varphi(r,\theta,\alpha)=R(r)\cdot Y(\theta)}

\fc{\Delta \varphi(r,\theta,\alpha)=0\Rightarrow \underset{=l(l+1)}{\frac{1}{R}\frac{d}{dr} \left( r^2 \frac{dR}{dr} \right)}+\underset{=-l(l+1)}{\frac{1}{Y \sin \theta}\frac{d}{d\theta} \left( \sin \theta \frac{dY}{d\theta} \right)}=0}

При \mdd{R(r)\varpropto r'}

или \mdd{R(r)\varpropto \frac{1}{r^{l+1}}}

\fc{\frac{1}{R}(r^2R')'=C} 

ищем решение в виде \mdd{R(r)\varpropto r^l}

\fc{\frac{1}{R}(r^2R')'=\underset{=-(l'+1)}{l}\cdot \underset{=(-l'-1+1)=(l'+1)l'}{(l+1)}} 

При этом \mdd{R(r)\varpropto \frac{1}{r^{l+1}}} удолетвор. уравнению с той же С 

(замена \mdd{l'=-(l+1)}) 

\qn{6. Уравнение Лапласа. Разделение переменных в уравнении Лапласа в
цилиндрической системе координат.}

\tc{Уравнение Лапласа(повтор)}

\tc{Разделение переменных в уравнении Лапласа в
цилиндрической системе координат}

Пусть \mdd{\varphi(r,\alpha)=\varphi(r,\alpha)}. Кроме того \mdd{\varphi(r,\alpha)=R(z)Y(\alpha)}
(то есть переменные разделяются)

\fc{Y(\alpha)=e^{\pm im\alpha}}

\fc{\varphi(r,\alpha)=R(r)(\underset{i}{\Sigma}e^{ im\alpha}),\text{где m}\in Z}

Пусть внутри, рассматриваемой области нет зарядов \mdd{\Rightarrow \Delta \varphi=0 \Rightarrow}

\fc{\Rightarrow \frac{1}{r} \frac{\partial}{\partial r}\left(r \frac{\partial \varphi}{\partial r}\right)+\frac{1}{r^2} \frac{\partial^2 \varphi}{\partial \alpha^2}=0 \Rightarrow e^{i m \alpha} \cdot \frac{1}{r}\left(r R^{\prime}\right)^{\prime}+R \cdot \frac{1}{r^2}\left(-m^2 e^{i m \alpha}\right)=0 \Rightarrow \frac{r(rR')'}{R}=m^2}

\newpage


Ищем решение в виде \mdd{R(r)\varpropto r^l}:
\fc{l^2=m^2 ,\text{т.е }l=\pm m \text{. Т.е } \varphi(r,\alpha)=\left(\frac{C_1}{r^m}+C_2r^m\right)e^{\pm im\alpha}}

\qn{7. Граничные условия для нормальной и тангенциальной компонент
электрического поля. Поверхностная плотность зарядов. Поле вблизи
поверхности металлов. Граничные условия для электрического поля,
выраженные через его скалярный потенциал.}

\tc{Граничные условия для нормальной и тангенциальной компонент
электрического поля}

\kr{Тангенциальная компонента}

\fc{\mm{rot}\vec{E}=0 \Rightarrow \oint \vec{E}d\vec{l}\leftarrow \text{интегральная форма}}

По теореме Стокса

\fc{0=\underset{(\forall )S}{\iint} \mm{rot}\vec{E}dS=\underset{(\forall )S}{\oint }\vec{E}d\vec{l}}

\imc[1.\textwidth]{10.png} 

\newpage


\fc{\oint \vec{E}d\vec{l}=E_x|_1\cdot l-E_x|_2\cdot l\Rightarrow E_x|_1=E_x|_2}

или же 

\fc{E_\tau|_1=E_\tau|_2}

\kr{Нормальная компонента}

\fc{\mm{div}\vec{E}=4\pi \rho \Rightarrow \underset{(\forall)S}{\oiint}\vec{E}d\vec{S}=4\pi Q \Rightarrow}

\fc{\Rightarrow \underset{(\forall)V}{\iiint} \mm{div}\vec{E}dV=4\pi \underset{(\forall)V}{\iiint} \rho dV \Rightarrow \underset{(\forall)\delta S}{\oiint}\vec{E}d\vec{S}=4\pi Q }

\imc[0.7\textwidth]{11.png}

\fc{E_{1n}|\cdot S-E_{2n}|\cdot S=4\pi Q=4\pi \rho S}

или же 

\fc{E_{1n}| -E_{2n}| =4\pi \rho }

\tc{Поверхностная плотность зарядов(???)}

\fc{dq\overset{df}{=}\sigma dS}

\newpage


\tc{Поле вблизи поверхности металлов}

Надо доказать что поле вблизи металлов равно

\fc{\vec{E}=4\pi \sigma \vec{n}}

Рассмотрим тангенсальную и нормальную компоненту поля \mdd{\vec{E}} на границе металла
 
\imc[0.7\textwidth]{12.png}

Если в проводнике имеется электрическое поле, то по нему течёт ток. Следовательно, для электростатических явлений электрическое поле внутри проводника \mdd{E_{1n}=0} отсюда 

\fc{E_{2n}=4\pi \sigma}

Снаружи металла поле \mdd{E_{2\tau }=0} и из граничных условий  

\fc{E_{1\tau}=0}

Итоговое поле равно 

\fc{\vec{E}_{2n}=4\pi \sigma \vec{n}}

Что и требовалось доказать.

\newpage


\tc{Граничные условия для электрического поля,
выраженные через его скалярный потенциал}

Можно рассмотреть две точки А и Б с одной стороны поверхности и В,Г с другой стороны. Найдем напряжение между парами этих  точек: \mdd{E_{\text{АБ}}| \text{ и } E_{\text{ВГ}}|}:

\imc[0.5\textwidth]{13.png}

Из граничных условий, что \mdd{E_{\tau}-}непрерывно следует, что:

\fc{E_{\text{АБ}|} = E_{\text{ВГ}}|} 

потенциал можно выразить через напряженность так:

\fc{{E}=-\mm{grad}\varphi}

отсюда получаем, что \mdd{\varphi_{\text{АБ}}|=\varphi_{\text{ВГ}}|\Rightarrow \varphi|-\text{непрерывно}}

\qn{8. Проводники в электрическом поле. Теорема единственности.}

\tc{Проводники в электрическом поле}


\end{document}