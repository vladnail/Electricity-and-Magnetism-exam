\section{Циркуляция и ротор электрического поля. Теорема Стокса. Электрический
потенциал. Работа электрического поля. Потенциал точечного заряда.}

\subsection*{Циркуляция и ротор электрического поля}

Циркуляция векторного поля $\vec{E}$ вдоль контура L

\fc{\underset{L}{\int} \vec{E} d\vec{l}} 

а по замкнутому контуру

\fc{\boxed{\underset{L}{\oint} \vec{E} d\vec{l}=0} }

или в дифференциальной форме 

\fc{\boxed{\mm{rot}\vec{E}=0}}

\imc[0.5\textwidth]{5.png} 

Как следствие из теоремы о циркуляции $\vec{E}$ работа при перемещении
заряда из одной точки поля в другую не зависит от формы траектории движения.

\subsection*{Теорема Стокса}

\fc{\boxed{\underset{\delta
S}{\oint}\vec{E}d\vec{l}=\underset{S}{\iint}\mm{rot}\vec{E}d\vec{S}}}

\imc[0.5\textwidth]{6.png}
 
\[\begin{array}{l|l}
    (\mm{rot}\vec{E})_{\vec{n}'}=\frac{\mm{rot}\vec{E}}{\frac{1}{\mm{cos}\theta}} & \Rightarrow (\mm{rot}\vec{E})_{\vec{n}'}=\mm{rot}\vec{E}\cdot\mm{cos}\theta\\
    \frac{1}{\mm{cos}\theta}=\frac{dS_{\vec{n}'}}{dS_{\vec{n}}} &
\end{array}
\]

\subsection*{Электрический потенциал}

Рассмотрим скаляроное поле

\imc[0.5\textwidth]{7.png}

\fc{\boxed{\varphi(\vec{r})\overset{df}{=}\int_{\vec{r}}^{\vec{r_0}}\vec{E}d\vec{l}}}

Чтобы определение было корректным, нужно чтобы этот интеграл не зависел от
формы $L$.

\kr{Доказательство:}

\fc{\mm{rot}\vec{E}=0 \Rightarrow \forall L
\underset{L}{\oint}\vec{E}d\vec{l}=0}

Запишем выражение при обходе $L-L'$ - сначала идем по контуру $L$, а
потом обратно по контуру $L'$ : 

\fc{\underset{L-L'}{\oint}\vec{E}d\vec{l}=0=\underset{L}{\oint}\vec{E}d\vec{l}
-\underset{L'}{\oint}\vec{E}d\vec{l}=0}

Такие поля называются потенциальными.

\kr{Доказано.}

Еще свойства потенциала: 

\imc[0.35\textwidth]{8.png}
   
\fc{\varphi(\vec{r})=\int \vec{E}(\vec{r})d\vec{r} + \underset{L}{\int}
\vec{E}d\vec{l} }

\fc{\text{и}}

\fc{\varphi(\vec{r}+d\vec{r_0})=\underset{L}{\int} \vec{E}d\vec{l}}
 
отсюда получим

\fc{\varphi(\vec{r}+d\vec{r_0})-\varphi(\vec{r})=-\vec{E}(\vec{r})d\vec{r}}

\fc{\text{так же используем}}

\fc{d\varphi=d\vec{r}\gradd \varphi }

отсюда получим

\fc{\forall d\vec{r} , \vec{E}(\vec{r})d\vec{r}=-\grad \varphi d\vec{r} \Rightarrow \vec{E}=-\mm{grad} \varphi=-\grad \varphi }

\fc{\mm{rot}\vec{E}=0=-[\grad \times \grad \varphi]}  

\subsection*{Работа электрического поля}

\fc{A=\underset{L}{\int}\vec{F}d\vec{l}=\underset{L}{\int}q\vec{E}d\vec{l}=q\left[\int_{\vec{r_1}}^{\vec{r_0}}\vec{E}d\vec{l}- \int_{\vec{r_2}}^{\vec{r_1}}\vec{E}d\vec{l}   \right]=q(\varphi_2-\varphi_2)=qU}

\subsection*{Потенциал точечного заряда}

\fc{\varphi(r) = \frac{q}{r}}

или в общем виде

\fc{\varphi(\vec{r}) = \underset{i}{\Sigma} \frac{q_i}{|\vec{r} - \vec{r}_i|}}